% Document setup
\documentclass[article, a4paper, 11pt, oneside]{memoir}
\usepackage[utf8]{inputenc}
\usepackage[T1]{fontenc}
\usepackage[UKenglish]{babel}

% Document info
\newcommand\doctitle{Separation Axioms}
\newcommand\docauthor{Danny Nygård Hansen}

% Formatting and layout
\usepackage[autostyle]{csquotes}
\usepackage[final]{microtype}
\usepackage{xcolor}
\frenchspacing
\usepackage{latex-sty/articlepagestyle}
\usepackage{latex-sty/articlesectionstyle}

% Fonts
\usepackage[largesmallcaps,partialup]{kpfonts}
\DeclareSymbolFontAlphabet{\mathrm}{operators} % https://tex.stackexchange.com/questions/40874/kpfonts-siunitx-and-math-alphabets
\linespread{1.06}
\let\mathfrak\undefined
\usepackage{eufrak}
\usepackage{inconsolata}
\usepackage{amssymb}

% Hyperlinks
\usepackage{hyperref}
\definecolor{linkcolor}{HTML}{4f4fa3}
\hypersetup{%
	pdftitle=\doctitle,
	pdfauthor=\docauthor,
	colorlinks,
	linkcolor=linkcolor,
	citecolor=linkcolor,
	urlcolor=linkcolor,
	bookmarksnumbered=true
}

% Equation numbering
\numberwithin{equation}{chapter}

% Footnotes
\footmarkstyle{\textsuperscript{#1}\hspace{0.25em}}

% Mathematics
\usepackage{latex-sty/basicmathcommands}
\usepackage{latex-sty/framedtheorems}
\usepackage{latex-sty/topologycommands}
\usepackage{tikz-cd}
\usetikzlibrary{babel}

% Lists
\usepackage{enumitem}
\setenumerate[0]{label=\normalfont(\arabic*)}

% Bibliography
\usepackage[backend=biber, style=authoryear, maxcitenames=2, useprefix]{biblatex}
\addbibresource{references.bib}

% Title
\title{\doctitle}
\author{\docauthor}


% Section style -- add to section style .sty?
\setsubsecheadstyle{\normalfont\itshape}


% Preimage -- to be added to mathcommands .sty
\newcommand{\preim}{^{-1}}


\newcommand{\calN}{\mathcal{N}}
\DeclarePairedDelimiter{\nhoodfilteraux}{(}{)}
% \newcommand{\nhoodfilter}[1]{\calN\nhoodfilteraux{#1}}
\newcommand{\nhoodfilter}[1]{\calN_{#1}}


\newcommand{\calU}{\mathcal{U}}
\newcommand{\calV}{\mathcal{V}}
\newcommand{\calW}{\mathcal{W}}
\newcommand{\calT}{\mathcal{T}}


\begin{document}

\maketitle

\chapter{Introduction}

In these notes we give an overview of some of the most important (and weakest) separation axioms in point-set topology. For each axiom we consider the following:

\begin{enumerate}
    \item What the axiom \emph{means} intuitively. Of course we understand what it means to \textquote{separate} points or sets in a topological space and how the ability to do so is useful. But many of the axioms are equivalent to properties of a space that (seemingly) have nothing to do with separation, and whose importance are perhaps more intuitively clear.

    \item How the axiom behaves with respect to topological constructions like subspaces and products.
    
    \item Other properties of the axioms. Each axiom interacts with the surrounding theory in different ways.
\end{enumerate}


\section{Overview of the axioms}

We here summarise the axioms by citing

\begin{itemize}
    \item $T_0$: The topology respects the underlying set structure by distinguishing between distinct points.
    
    \item $T_1$: Singletons are closed, or in other words, if $(x_\alpha)_{\alpha \in A}$ is a constant net with $x_\alpha = x$ for all $\alpha \in A$, then $x$ is the unique limit of $(x_\alpha)$.
    
    \item Hausdorff: All limits are unique (\cref{thm:Hausdorff-equivalent-properties}).
    
    \item Regularity: Every open neighbourhood contains a closed neighbourhood (\cref{thm:regular-equivalent-properties}).
    
    \item Complete regularity: Has enough bounded continuous functions to determine its topology (\cref{thm:completely-regular-weak-topology}).
    
    \item Normality: Continuous functions on closed sets can be extended to the entire space (\cref{thm:Tietze-extension}). Why should we restrict to \emph{closed} sets? Continuous functions defined on a non-closed set may not even extend to a continuous function on the \emph{closure} of that set: Take for instance the function $x \mapsto 1/x$ on $\reals \setminus \{0\}$.
\end{itemize}


\chapter{Preliminary definitions and results}

If $X$ is a topological space and $A \subseteq X$, then we say that a set $N \subseteq X$ is a \emph{neighbourhood} of $A$ if there is an open set $U$ in $X$ such that $A \subseteq U \subseteq N$. The family of neighbourhoods of a set $A$ is called the \emph{neighbourhood filter of $A$} and is denoted $\nhoodfilter{A}$. If $A = \{x\}$ is a singleton we also write $\nhoodfilter{x}$ and call $N$ a neighbourhood of $x$.

We begin by proving a series of results that do not pertain directly to the separability axioms, but that will be used in the sequel.

If $X$ is a topological space, then we say that two disjoint subsets $A$ and $B$ can be separated if there exist disjoint open sets $U$ and $V$ with $A \subseteq U$ and $B \subseteq V$.

\begin{lemma}
    \label{thm:separating_from_compacts}
    Let $X$ be a topological space, and let $A \subseteq X$ be a set that can be separated from points that lie in a set $B \subseteq X$. Then $A$ can be separated from compact sets contained in $B$.
\end{lemma}
%
This is an example of the general principle that compact sets often act like points.

\begin{proof}
    Let $K \subseteq X$ be a compact set contained in $B$. Since $A$ can be separated from points in $B$, then for every $x \in K$ there are disjoint open sets $U_x$ and $V_x$ such that $x \in U_x$ and $A \subseteq V_x$. The collection $(U_x)_{x \in K}$ is an open cover of $K$, so there is a finite subcover $(U_{x_i})_{i=1}^n$. Let $U = \bigunion_{i=1}^n U_{x_i}$ and $V = \bigintersect_{i=1}^n V_{x_i}$. Then $U$ and $V$ are disjoint open sets containing $K$ and $A$ respectively.
\end{proof}


\begin{lemma}
    Let $(S,\rho)$ be a pseudometric space, and let $A \subseteq S$ be nonempty. Define a map $\rho(\,\cdot\,, A) \colon S \to [0, \infty)$ by
    %
    \begin{equation*}
        \rho(x, A) = \inf_{a \in A} \rho(x,a).
    \end{equation*}
    %
    This map has the following properties:
    %
    \begin{enumlem}
        \item \label{enum:distance-to-set-closure} $\rho(x,A) = 0$ if and only if $x \in \closure{A}$.

        \item \label{enum:distance-to-set-triangle-inequality} If $y \in S$, then $\rho(x,A) \leq \rho(x,y) + \rho(y,A)$.

        \item \label{enum:distance-to-set-continuous} $\rho(\,\cdot\,, A)$ is continuous.
    \end{enumlem}
\end{lemma}

\begin{proof}
    First we prove \subcref{enum:distance-to-set-closure}. Notice that $\rho(x,A) = 0$ if and only if for any $r > 0$ there is an $a \in A$ such that $\rho(x,a) < r$. But this is true precisely when any ball $B(x,r)$ intersects $A$, i.e. when $x \in \closure{A}$.

    For any $a \in A$ we have
    %
    \begin{equation*}
        \rho(x,A)
            \leq \rho(x,a)
            \leq \rho(x,y) + \rho(y,a),
    \end{equation*}
    %
    and since this is true for any $a \in A$, \subcref{enum:distance-to-set-triangle-inequality} follows.

    Finally, \subcref{enum:distance-to-set-continuous} follows immediately from (ii), since
    %
    \begin{equation*}
        \rho(x,A) - \rho(y,A)
            \leq \rho(x,y)
    \end{equation*}
    %
    for all $x,y \in S$.
\end{proof}

\newcommand{\calA}{\mathcal{A}}

If $X$ is a topological space, we say that a collection $\calU$ of subsets of $X$ is a \emph{cover} of $X$ if $X = \bigunion_{U \in \calU} U$. A \emph{subcover} of a cover $\calU$ is a subcollection of $\calU$ that itself is a cover. An \emph{open cover} is a cover consisting of open sets.

If $\calU$ and $\calV$ are covers of $X$, we say that $\calU$ \emph{refines} $\calV$ and is a \emph{refinement} of $V$ if each $U \in \calU$ is contained in some $V \in \calV$. Notice that a subcover is in particular a refinement.

A collection $\calA$ of subsets of $X$ is called \emph{locally finite} if every point of $X$ has a neighbourhood that intersects finitely many elements in $\calA$. It is easy to show that if $\calA$ is locally finite, then $\closure{\calA} = \set{\closure{A}}{A \in \calA}$ is also locally finite.

\begin{definition}
    Let $X$ be a topological space.
    %
    \begin{enumdef}
        \item $X$ is \emph{compact} if every open cover of $X$ has a finite subcover.

        \item $X$ is \emph{Lindelöf} if every open cover of $X$ has a countable subcover.

        \item $X$ is \emph{paracompact} if every open cover of $X$ has an open locally finite refinement.
    \end{enumdef}
\end{definition}
%
Clearly a compact space is both Lindelöf and paracompact.


\begin{proposition}
    \label{thm:compact-Lindelof-closed-subset}
    Let $X$ be a topological space. If $X$ is compact/Lindelöf, then every closed subset of $X$ is compact/Lindelöf in the subspace topology.
\end{proposition}
%
This result is of course standard for compact spaces, but we include the proof to illustrate that it is identical to the proof for the Lindelöf case.

\begin{proof}
    Let $X$ be compact/Lindelöf, and let $A \subseteq X$ be closed. If $\calU$ is an open cover of $A$, then there is a collection $\calV$ of sets open on $X$ such that $\calU = \set{V \intersect A}{V \in \calV}$. By adjoining $A^c$ to $\calV$ we obtain an open cover of $X$, so it has a finite/countable subcover $\calV'$. Discarding $A^c$ and intersecting every other set in $\calV'$ with $A$ yields a finite/countable subcover of $\calU$, proving the claim.
\end{proof}


\begin{proposition}
    The continuous image of a compact/Lindelöf space is compact/Lindelöf.
\end{proposition}

\begin{proof}
    Let $f \colon X \to Y$ be continuous and surjective, and assume that $X$ is compact/Lindelöf. Let $\calU$ be an open cover of $Y$. Then $f\preim(\calU)$ is an open cover of $X$, so it has a finite/countable subcover $f\preim(\calU')$. But then $\calU'$ is a finite/countable subcover of $\calU$ as desired.
\end{proof}



\chapter[The T0 axiom][The $T_0$ axiom]{The $T_0$ axiom}

\section{Definition and equivalent properties}

\begin{definition}
    A topological space $X$ satisfies the $T_0$ axiom and is called a \emph{$T_0$-space} if, for every pair of distinct points $x,y \in X$, either $x$ has a neighbourhood that does not contain $y$, or vice versa.
\end{definition}

We begin by giving an alternative characterisation of $T_0$-spaces: Let $X$ be a topological space. We define an ordering on $X$ called the \emph{specialisation preorder} by letting $x \leq y$ if $x \in \closure{\{y\}}$ for $x,y \in X$. It is clear that $\leq$ is in fact a preorder, and so it determines an equivalence relation $\equiv$; that is, $x \equiv y$ if and only if $x \leq y$ and $y \leq x$.

It is easy to show that $x \leq y$ if and only if $\nhoodfilter{x} \subseteq \nhoodfilter{y}$.  If $x \equiv y$, then we say that $x$ and $y$ are \emph{topologically indistinguishable} since then $x$ and $y$ have the same neighbourhoods.

It is clear that $X$ is $T_0$ if and only if the relation $\equiv$ is trivial, that is if $x$ and $y$ are topologically indistinguishable precisely when $x = y$. The quotient space $X/{\equiv}$ is called the \emph{$T_0$-identification} or the \emph{Kolmogorov quotient} of $X$, and it is indeed $T_0$:

\begin{proposition}[The $T_0$-identification]
    Let $X$ be a topological space, and let $q \colon X \to X/{\equiv}$ be the quotient map onto the $T_0$-identification of $X$. Then
    %
    \begin{enumprop}
        \item $q$ is an open and closed map,
        \item $X/{\equiv}$ is $T_0$, and
        \item \label{enum:T0-identification-is-smallest-equiv} if $\sim$ is an equivalence relation on $X$ such that $X/{\sim}$ is $T_0$, then ${\equiv} \subseteq {\sim}$.\footnotemark
    \end{enumprop}
\end{proposition}
\footnotetext{Both $\equiv$ and $\sim$ are subsets of $X \times X$, so this inclusion means that if $x \equiv y$ then $x \sim y$ for all $x,y \in X$.}
%
Part \subcref{enum:T0-identification-is-smallest-equiv} expresses the fact that $\equiv$ is the most conservative equivalence relation on $X$ that makes the corresponding quotient a $T_0$-space.

\begin{proof}
    We first show that all open sets are saturated. Let $U$ be an open set of $X$, and let $x \in U$. If $x \equiv x'$, then $U$ is also a neighbourhood of $x'$, so $x' \in U$. It follows that $q(x) \subseteq U$, and so
    %
    \begin{equation*}
        U = \bigunion_{x \in U} q(x).
    \end{equation*}
    %
    Hence $U$ is a union of $\equiv$-equivalence classes, and thus it is saturated. The complement $U^c$ is the union of all equivalence classes $q(x)$ for $x \in U^c$, so closed sets are also saturated. It follows that $q(U)$ is open and $q(U^c)$ is closed, so $q$ is an open and closed map.

    Now we show that $X/{\equiv}$ is $T_0$. Assume that $x \not\equiv y$. Without loss of generality we may assume the existence of an element $U \in \nhoodfilter{x} \setminus \nhoodfilter{y}$. Since $q$ is open, $q(U)$ is a neighbourhood of $q(x)$ in $X/{\equiv}$. We claim that $q(y) \not\in q(U)$: If not, then since $U$ is a union of equivalence classes there is a $z \in U$ with $y \equiv z$. But then $y \in U$ which is a contradiction. Thus $X/{\equiv}$ is $T_0$.

    Finally, let $X/{\sim}$ be $T_0$, and let $p \colon X \to X/{\sim}$ be the quotient map. If $x \not\sim y$ then $p(x) \neq p(y)$, so without loss of generality we may choose an open set $U \subseteq X/{\sim}$ with $p(x) \in U$ and $p(y) \not\in U$. Then $x \in p\preim(U)$ and $y \not\in p\preim(U)$, so $p\preim(U)$ is a neighbourhood of $x$ that is not a neighbourhood of $y$, so $x \not\equiv y$.
\end{proof}

We explore the $T_0$-identification in the context of metric spaces: Let $(S,\rho)$ be a pseudometric space, and define a relation $\sim$ on $S$ by $x \sim y$ if and only if $\rho(x,y) = 0$. This is clearly an equivalence relation. Let $\tilde{S} = S/{\sim}$ and define a map $\tilde{\rho} \colon \tilde{S} \prod \tilde{S} \to [0,\infty)$ by $\tilde{\rho}([x],[y]) = \rho(x,y)$. This is well-defined, since if $x \sim x'$ and $y \sim y'$ then
%
\begin{equation*}
    \rho(x,y)
        \leq \rho(x,x') + \rho(x',y') + \rho(y',y)
        = \rho(x',y').
\end{equation*}
%
It is obvious that $\tilde{\rho}$ is then a metric on $\tilde{S}$, and we call $(\tilde{S}, \tilde{\rho})$ the \emph{metric identification} of $(S,\rho)$.

\begin{proposition}
    Let $(S,\rho)$ be a pseudometric space. Then the relation $\sim$ defined above and the topological indistinguishability relation $\equiv$ coincide.
\end{proposition}

\begin{proof}
    It suffices to show that $x \sim y$ if and only if $\nhoodfilter{x} = \nhoodfilter{y}$ for all $x,y \in S$. Assume that $x \sim y$ and let $U \in \nhoodfilter{x}$. Then there is an $r > 0$ such that $B(x,r) \subseteq U$. But we clearly have $y \in B(x,r)$, so $U \in \nhoodfilter{y}$. This shows that $\nhoodfilter{x} \subseteq \nhoodfilter{y}$, and the opposite inclusion follows by symmetry.

    Conversely, assume that $x \not\sim y$. Then $0 < r < \rho(x,y)$ for some $r$, and $B(x,r)$ is a neighbourhood of $x$ but not of $y$.
\end{proof}


\section[Operations on T0-spaces][Operations on $T_0$-spaces]{Operations on $T_0$-spaces}


\begin{proposition}
    \label{thm:T0-properties}
    \begin{enumprop}
        \item \label{enum:T0-subspace} Any subspace of a $T_0$-space is $T_0$.
        \item A nonempty product space is $T_0$ if and only if every factor is $T_0$.
    \end{enumprop}
\end{proposition}

\begin{proof}
    Let $X$ be a $T_0$-space and $A \subseteq X$. If $x,y \in A$ and $x \neq y$, then without loss of generality we may choose a neighbourhood $U$ of $x$ in $X$ that does not contain $Y$. But then $U \intersect A$ is a neighbourhood of $x$ in $A$ that doesn't contain $y$, so $A$ is $T_0$.

    Now let $(X_\alpha)_{\alpha \in A}$ be a collection of topological spaces, and assume that the product $X = \bigprod_{\alpha \in A} X_\alpha$ is nonempty. Assume that all $X_\alpha$ are $T_0$, and let $x,y \in X$ be distinct points. Then there is a $\beta \in A$ such that $x_\beta \neq y_\beta$. Without loss of generality we may pick a neighbourhood $U$ of $x_\beta$ in $X_\beta$ that does not contain $y_\beta$. Then $\pi_\beta\preim(U)$ is a neighbourhood of $x$ that does not contain $y$.

    Conversely, assume that the product $X$ is $T_0$ and let $\beta \in A$. Pick a point $y \in X$, and let
    %
    \begin{equation*}
        Y
            = \set{x \in X}{x_\alpha = y_\alpha \text{ for } \alpha \neq \beta}.
    \end{equation*}
    %
    Then $Y$ is $T_0$ by \subcref{enum:T0-subspace}, so $X_\beta \cong Y$ is also $T_0$.
\end{proof}



\chapter[The T1 axiom][The $T_1$ axiom]{The $T_1$ axiom}

\section{Definition and equivalent properties}

\begin{definition}
    A topological space $X$ satisfies the $T_1$ axiom and is called a \emph{$T_1$-space} if, for every pair of distinct points $x,y \in X$, $x$ has a neighbourhood disjoint from $y$, and $y$ has a neighbourhood disjoint from $x$.
\end{definition}
%
Clearly every $T_1$-space is $T_0$. We begin by giving some properties of topological spaces that are equivalent to the $T_1$ axiom:

\begin{proposition}
    The following are equivalent for a topological space $X$:
    %
    \begin{enumprop}
        \item \label{enum:T1-space} $X$ is $T_1$,
        \item \label{enum:T1-singletons-closed} each singleton of $X$ is closed, and
        \item \label{enum:T1-intersection-of-open-sets} each subset of $X$ is the intersection of all open sets containing it.
    \end{enumprop}
\end{proposition}

\begin{proof}
    \subcref{enum:T1-space} $\implies$ \subcref{enum:T1-singletons-closed}: If $X$ is $T_1$ and $x \in X$, then every point $y \in X \setminus \{x\}$ has a neighbourhood disjoint from $\{x\}$ so $X \setminus \{x\}$ is open.

    \subcref{enum:T1-singletons-closed} $\implies$ \subcref{enum:T1-intersection-of-open-sets}: If $A \subseteq X$, then
    %
    \begin{equation*}
        A = \bigintersect_{x \not\in A} X \setminus \{x\},
    \end{equation*}
    %
    so $A$ is an intersection of open sets.

    \subcref{enum:T1-intersection-of-open-sets} $\implies$ \subcref{enum:T1-space}: If $x,y \in X$ with $x \neq y$, then there is an open subset containing $x$ and not $y$, and vice versa.
\end{proof}


\section[Operations on T1-spaces][Operations on $T_1$-spaces]{Operations on $T_1$-spaces}

\begin{proposition}
    \label{thm:T1-properties}
    \begin{enumprop}
        \item \label{enum:T1-subspace} Any subspace of a $T_1$-space is $T_1$.
        \item A nonempty product space is $T_1$ if and only if every factor is $T_1$.
        \item \label{enum:T1-quotient} A quotient space $X/{\sim}$ is $T_1$ if and only if every $\sim$-equivalence class is closed in $X$.
    \end{enumprop}
\end{proposition}

\begin{proof}
    The proof of the first two claims are almost identical to the proof of \cref{thm:T0-properties}, so we omit it.

    We prove \subcref{enum:T1-quotient}: Fibres of the quotient map are precisely the equivalence classes, so by the definition of the quotient topology, singletons of $X/{\sim}$ are closed if and only if the corresponding equivalence class is closed as a subset of $X$.
\end{proof}


\section[Conditions for the T1 axiom]{Conditions for the $T_1$ axiom}

\begin{proposition}
    The closed image\footnotemark{} of a $T_1$-space is $T_1$.
\end{proposition}
\footnotetext{By \textquote{closed image} we mean the image of a closed (not necessarily continuous) map.}

\begin{proof}
    Let $f \colon X \to Y$ be a closed map from a $T_1$-space $X$ to a topological space $Y$, and let $y \in f(X)$. Then there is some $x \in X$ with $f(x) = y$, and since $\{x\}$ is closed in $X$ and $f$ is closed, $\{y\}$ is closed in $Y$ and hence in $f(X)$.
\end{proof}



\chapter{Hausdorff spaces}

\section{Definition and equivalent properties}

\begin{definition}
    A topological space $X$ satisfies the $T_2$ axiom and is called a \emph{$T_2$-space} or \emph{Hausdorff space} if, for every pair of distinct points $x,y \in X$, $x$ has a neighbourhood $U$ and $y$ a neighbourhood $V$ with $U \intersect V = \emptyset$.
\end{definition}
%
Again, a $T_2$-space is clearly $T_1$. We give a series of conditions that are equivalent to the $T_2$ axiom.


\begin{proposition}
    \label{thm:Hausdorff-equivalent-properties}
    The following are equivalent for a topological space $X$:
    %
    \begin{enumprop}
        \item \label{enum:T2-space} $X$ is Hausdorff,
        \item \label{enum:T2-limits_unique} limits of nets (and hence of filters) in $X$ are unique, and
        \item \label{enum:T2-closed_diagonal} the diagonal $\Delta = \set{(x,x)}{x \in X}$ is closed in $X \prod X$.
    \end{enumprop}
\end{proposition}

\begin{proof}
    \subcref{enum:T2-space} $\implies$ \subcref{enum:T2-limits_unique}: Let $(x_\alpha)_{\alpha \in A}$ be a net in $X$, and assume that $x_\alpha \to x$ and $x_\alpha \to y$. For every pair of neighbourhoods $U$ of $x$ and $V$ of $y$, $(x_\alpha)$ is eventually in $U \intersect V$. Hence $x$ and $y$ have no pair of disjoint neighbourhoods, so $x = y$.

    \subcref{enum:T2-limits_unique} $\implies$ \subcref{enum:T2-closed_diagonal}: If $\Delta$ were not closed, then there would exist a net $(x_\alpha)_{\alpha \in A}$ in $X$ such that $(x_\alpha, x_\alpha) \to (x,y)$ where $x \neq y$, so the limit of $(x_\alpha)$ would not be unique.

    \subcref{enum:T2-closed_diagonal} $\implies$ \subcref{enum:T2-space}: Let $x,y \in X$ be distinct points so that $(x,y) \not\in \Delta$. If $\Delta$ is closed, then $(x,y)$ has a neighbourhood $U \prod V$ in $X \times X$ disjoint from $\Delta$. But then $U$ and $V$ are disjoint neighbourhoods of $x$ and $y$ respectively, so $X$ is Hausdorff.
\end{proof}


\section{Further properties of Hausdorff spaces}

\begin{proposition}
    In a Hausdorff space, disjoint compact sets can be separated.
\end{proposition}

\begin{proof}
    Let $K_1$ and $K_2$ be disjoint compact sets in a Hausdorff space $X$, and fix a point $x \in K_1$. Since $X$ is Hausdorff, $x$ can be separated from every $y \in K_2$. It follows from \cref{thm:separating_from_compacts} that $x$ can be separated from $K_2$. But then $K_2$ can be separated from every point in $K_1$, so another application of \cref{thm:separating_from_compacts} yields the desired claim.
\end{proof}


\begin{proposition}
    If $f,g \colon X \to Y$ are continuous and $Y$ is Hausdorff, then the set $\{f = g\} = \set{x \in X}{f(x) = g(x)}$ is closed. In particular, if $f$ and $g$ agree on a dense subset of $X$, then $f = g$.
\end{proposition}

\begin{proof}
    Let $x \in X$ be such that $f(x) \neq g(x)$. Since $Y$ is Hausdorff, $f(x)$ and $g(x)$ have disjoint neighbourhoods $U$ and $V$ respectively. Then $f\preim(U) \intersect g\preim(V)$ is a neighbourhood of $x$ on which $f$ and $g$ differ. Thus $\{f \neq g\}$ is open which proves the claim.

    Alternatively we may argue using nets\footnote{We refrain from using nets (or filters) as far as possible, or at least also provide proofs that do not depend on them. In this case nets do in fact clarify the necessity of the Hausdorff assumption, so we also include a proof using nets.}: Let $(x_\alpha)_{\alpha \in A}$ be a net in $\{f = g\}$ such that $x_\alpha \to x$. By continuity we have $f(x_\alpha) \to f(x)$ and $g(x_\alpha) \to g(x)$, and since limits are unique in $Y$ by \cref{enum:T2-limits_unique} and $f(x_\alpha) = g(x_\alpha)$ we have $f(x) = g(x)$.
\end{proof}



\chapter[Regular and T3-spaces][Regular and $T_3$-spaces]{Regular and $T_3$-spaces}

\section{Definition and equivalent properties}

\begin{definition}
    A topological space $X$ is \emph{regular} if, for every point $x \in X$ and closed subset $A \subseteq X$ with $x \not\in A$, $x$ has a neighbourhood $U$ and $A$ a neighbourhood $V$ with $U \intersect V = \emptyset$.

    If furthermore $X$ is $T_1$, then $X$ is said to satisfy the $T_3$ axiom and is called a \emph{$T_3$-space}.
\end{definition}
%
Notice that a regular space is \emph{not} necessarily Hausdorff since singletons are not closed. Of course a $T_3$-space is Hausdorff.


\begin{proposition}
    \label{thm:regular-equivalent-properties}
    The following are equivalent for a topological space $X$:
    %
    \begin{enumprop}
        \item \label{enum:regular-space} $X$ is regular,

        \item \label{enum:closed-nhood-inside-nhood} if $U$ is an open neighbourhood of $x \in X$, then $x$ has a closed neighbourhood contained in $U$, and

        \item \label{enum:nhood-basis-of-closed-sets} every $x \in X$ has a neighbourhood basis of closed sets.
    \end{enumprop}
\end{proposition}

\begin{proof}
    \subcref{enum:regular-space} $\implies$ \subcref{enum:closed-nhood-inside-nhood}: Assume that $X$ is regular, and let $U$ be an open neighbourhood of $x \in X$. Then $U^c$ is closed, so there exist disjoint open sets $V$ and $W$ with $x \in V$ and $U^c \subseteq W$. Then $x \in V \subseteq W^c \subseteq U$, so $W^c$ is the desired closed neighbourhood.

    \subcref{enum:closed-nhood-inside-nhood} $\implies$ \subcref{enum:nhood-basis-of-closed-sets}: This is obvious.

    \subcref{enum:nhood-basis-of-closed-sets} $\implies$ \subcref{enum:regular-space}: Let $x \in X$ and $A \subseteq X$ closed with $x \not\in A$. Then $A^c$ is an open neighbourhood of $x$, so if \subcref{enum:nhood-basis-of-closed-sets} applies then $A^c$ contains a closed neighbourhood $B$ of $X$. Then $\interior{B}$ and $B^c$ are disjoint open neighbourhoods of $x$ and $A$ respectively.
\end{proof}


\section{Further properties of regular spaces}

\begin{proposition}
    In a regular space, compact sets can be separated from disjoint closed sets.
\end{proposition}

\begin{proof}
    This is a direct consequence of \cref{thm:separating_from_compacts}.
\end{proof}


\begin{proposition}
    \label{thm:regular-Lindelof-is-paracompact}
    A regular Lindelöf space is paracompact.
\end{proposition}

\begin{proof}
    Let $X$ be a regular Lindelöf space, and let $\calU$ be an open cover of $X$. For every $x \in X$ pick a $U_x \in \calU$ with $x \in U_x$. By regularity $x$ has a neighbourhood $V_x$ such that $\closure{V}_{\!\!x} \subseteq U_x$. Then $\set{V_x}{x \in X}$ is also an open cover of $X$, so it has countable subcover $\set{V_{x_n}}{n \in \naturals}$ since $X$ is Lindelöf.

    For $n \in \naturals$ define sets
    %
    \begin{equation*}
        W_n = U_{x_n} \setminus \bigunion_{k < n} \closure{V}_{\!\!x_k}.
    \end{equation*}
    %
    For $x \in X$ there is a smallest $k \in \naturals$ such that $x \in \closure{V}_{\!\!x_k}$, so $x \in W_k$. Hence $\set{W_n}{n \in \naturals}$ is also an open cover of $X$, and it is clearly a refinement of $\calU$. It is also locally finite, since $x \in V_{x_k}$ for some $k \in \naturals$, but $V_{x_k}$ does not intersect $W_n$ for $n > k$. Hence $X$ is paracompact.
\end{proof}


\section{Conditions for regularity}

\begin{corollary}
    Pseudometric spaces are regular.
\end{corollary}

\begin{proof}
    This will follow from \cref{thm:pseudometric-completely-regular} since completely regular spaces are regular.
\end{proof}



\chapter{Completely regular and Tychonoff spaces}

\section{Definition}

\begin{definition}
    A topological space $X$ is \emph{completely regular} if, for every point $x \in X$ and closed subset $A \subseteq X$ with $x \not\in A$, there is a continuous function $f \colon X \to [0,1]$ with $f(x) = 0$ and $f(A) = 1$. Such a function is said to \emph{separate} $x$ and $A$.

    If furthermore $X$ is $T_1$, then $X$ is said to satisfy the $T_{3\frac{1}{2}}$-axiom and is called \emph{Tychonoff}.
\end{definition}


If $X$ is a topological space, then we denote by $C(X)$ the space of continuous real-valued functions on $X$. The subspace of $C(X)$ consisting of bounded functions is denoted $C_b(X)$. We now prove that a space is completely regular precisely when the bounded continuous functions on the space induce the topology. Of course, these functions are already continuous, so this says that there are \emph{enough} continuous functions for them to characterise the topology.

To prove this we take a small detour by studying the defining property of completely regular spaces in greater generality. We say that a collection $(f_\alpha)_{\alpha \in A}$ of functions $f_\alpha \colon X \to X_\alpha$ between topological spaces \emph{separates points from closed sets} if whenever $C \subseteq X$ is closed and $x \not\in C$, then $f_\alpha(x) \not\in \closure{f_\alpha(C)}$ for some $\alpha \in A$.

\begin{proposition}
    \label{thm:separating-points-from-closed-sets-basis}
    A collection $(f_\alpha)_{\alpha \in A}$ of functions $f_\alpha \colon X \to X_\alpha$ between topological spaces separates points from closed sets if and only if the sets $f_\alpha\preim(V)$, for $\alpha \in A$ and $V \subseteq X_\alpha$ open, form a basis for the topology on $X$.
\end{proposition}

\begin{proof}
    First assume that $(f_\alpha)$ separates points from closed sets, let $U \subseteq X$ be open and let $x \in U$. Then $U^c$ is closed, so there is some $\alpha \in A$ such that $f_\alpha(x) \not\in \closure{f_\alpha(U^c)}$. Then
    %
    \begin{equation*}
        U^c
            \subseteq f_\alpha\preim \bigl( f_\alpha(U^c) \bigr)
            \subseteq f_\alpha\preim \bigl( \closure{f_\alpha(U^c)} \bigr).
    \end{equation*}
    %
    So letting $V = \closure{f_\alpha(U^c)}^c$ we find that $x \in f_\alpha\preim(V) \subseteq U$ as desired.

    Conversely, assume that the sets $f_\alpha\preim(V)$ form a basis for the topology on $X$. Let $x \in X$ and $C \subseteq X$ closed with $x \not\in C$. There is an $\alpha \in A$ and an open $V \subseteq X_\alpha$ such that $x \in f_\alpha\preim(V) \subseteq C^c$. Then $V$ is a neighbourhood of $f_\alpha(x)$ disjoint from $f_\alpha(C)$, so $f_\alpha(x) \not\in \closure{f_\alpha(C)}$.
\end{proof}


\begin{corollary}
    \label{thm:separating-points-from-closed-sets-weak-topology}
    If $(f_\alpha)_{\alpha \in A}$ is a collection of functions $f_\alpha \colon X \to X_\alpha$ between topological spaces which separates points from closed sets, then $X$ carries the weak topology induced by the maps $f_\alpha$.
\end{corollary}

\begin{proof}
    \cref{thm:separating-points-from-closed-sets-basis} shows that the collection of preimages $f_\alpha\preim(V)$, for $\alpha \in A$ and $V \subseteq X_\alpha$ open, forms a basis for the topology on $X$, so it in particular generates the topology.
\end{proof}



\begin{theorem}
    \label{thm:completely-regular-weak-topology}
    A topological space $X$ is completely regular if and only if it has the weak topology induced by $C_b(X)$.
\end{theorem}

\begin{proof}
    If $X$ is completely regular, then $C_b(X)$ separates points from closed sets by definition, so \cref{thm:separating-points-from-closed-sets-weak-topology} shows that $X$ carries the the weak topology induced by $C_b(X)$.

    Conversely, suppose that $X$ has the weak topology induced by $C_b(X)$. Let $U \subseteq X$ be open, and let $x \in U$. Then there are functions $f_1, \ldots, f_n \in C_b(X)$ and subbasic open sets $V_1, \ldots, V_n \subseteq \reals$ such that
    %
    \begin{equation*}
        x
            \in \bigintersect_{i=1}^n f_i\preim(V_i)
            \subseteq U.
    \end{equation*}
    %
    By changing the sign on the $f_i$ if necessary, we may assume that each $V_i$ is on the form $(a_i, \infty)$. Define functions $g_i \colon X \to \reals$ by $g_i(x) = (f_i(x) - a_i) \join 0$. Then $g_i\preim(0,\infty) = f_i\preim(a_i,\infty)$, so
    %
    \begin{equation*}
        x
            \in \bigintersect_{i=1}^n g_i\preim(0,\infty)
            \subseteq U.
    \end{equation*}
    %
    %
    Let $g = g_1 g_2 \cdots g_n$. Then $g(x) > 0$, so $x \in g\preim(0,\infty)$. Furthermore, if $g(y) > 0$ then each $g_i(y) > 0$. it follows that
    %
    \begin{equation*}
        x
            \in g\preim(0,\infty)
            \subseteq U.
    \end{equation*}
    %
    Then $g(x) \neq 0$, but $g(U^c) = 0$, so $X$ is completely regular.
\end{proof}


% \begin{proposition}
%     A topological space is Tychonoff if and only if it is homeomorphic to a subspace of a cube, i.e. a product of compact intervals.
% \end{proposition} % Leave this out?

% \begin{proof}
    
% \end{proof}


\section{Conditions for complete regularity}

\begin{proposition}
    \label{thm:pseudometric-completely-regular}
    Pseudometric spaces are completely regular.
\end{proposition}
%
In \cref{thm:metric-space-normal} we will see that pseudometric spaces are also normal, but since a pseudometric space is not necessarily $T_1$, this does not imply that it is (completely) regular. Hence the necessity of the present proposition.

\begin{proof}
    Let $(S,\rho)$ be a pseudometric space, $x \in S$, and let $A \subseteq S$ be closed with $x \not\in A$. Since $A$ is closed, the map $y \mapsto \rho(y,A)$ is zero on $A$ and nonzero at $y$.
\end{proof}


We now wish to show that locally compact Hausdorff spaces are completely regular. In the presence of the Hausdorff axiom, complete regularity is weaker than normality, and \emph{compact} spaces are normal, so it is perhaps not surprising that \emph{locally} compact spaces are completely regular.

To show this we will prove a version of Urysohn's lemma for locally compact Hausdorff spaces. This relies on the Urysohn lemma for normal spaces covered in the next section, but we place this discussion here since we are interested in it in the context of completely regular spaces.


\begin{lemma}
    Let $X$ be a locally compact Hausdorff space. If $K \subseteq U \subseteq X$ with $K$ compact and $U$ open, then there is a precompact open set $V$ with $K \subseteq V \subseteq \closure{V} \subseteq U$.
\end{lemma}

\begin{proof}
    
\end{proof}


\begin{theorem}[Urysohn's Lemma, locally compact version]
    Let $X$ be a locally compact Hausdorff space, and let $K \subseteq U \subseteq X$ with $K$ compact and $U$ open. Then there exists a continuous function $f \colon X \to [0,1]$ such that $f(K) = 1$ and $f$ vanishes outside a compact subset of $U$.
\end{theorem}

\begin{proof}
    By [lemma] there is a precompact open set $V$ with $K \subseteq V \subseteq \closure{V} \subseteq U$. Since compact Hausdorff spaces are normal, we can apply Urysohn's lemma for normal spaces to $\closure{V}$: This yields a continuous function $f \colon \closure{V} \to [0,1]$ with $f(K) = 1$ and $f(\boundary V) = 0$. Extend $f$ to $X$ by letting $f(\closure{V}^c) = 1$.

    We claim that $f$ is continuous on $X$. Let $B \subseteq [0,1]$ be closed. If $0 \not\in B$, then $f\preim(B) = (f|_{\closure{V}})\preim(B)$ is closed in $\closure{V}$, hence also in $X$. On the other hand, if $0 \in B$, then
    %
    \begin{equation*}
        f\preim(B)
            = (f|_{\closure{V}})\preim(B) \union \closure{V}^c
            = (f|_{\closure{V}})\preim(B) \union V^c,
    \end{equation*}
    %
    where the last equality follows since $\boundary V \subseteq (f|_{\closure{V}})\preim(B)$. Again $f\preim(B)$ is closed, so $f$ is continuous.
\end{proof}


\begin{corollary}
    Locally compact Hausdorff spaces are completely regular, hence Tychonoff.
\end{corollary}

\begin{proof}
    Let $X$ be a locally compact Hausdorff space, let $x \in X$ and $A \subseteq X$ be a closed subset. In the notation of Urysohn's lemma, let $K = \{x\}$ and $U = A^c$, which yields a continuous function $f \colon X \to [0,1]$ with $f(x) = 1$ and $f(A) = 0$.
\end{proof}



\chapter[Normal and T4-spaces]{Normal and $T_4$-spaces}

\section{Definition}

\begin{definition}
    A topological space $X$ is \emph{normal} if, for every pair of disjoint closed subsets $A,B \subseteq X$, $A$ has a neighbourhood $U$ and $B$ a neighbourhood $V$ with $U \intersect V = \emptyset$.

    If furthermore $X$ is $T_1$, then $X$ is said to satisfy the $T_4$ axiom and is called a \emph{$T_4$-space}.
\end{definition}

We discuss conditions that are equivalent to normality in our discussion of Urysohn's lemma below.


\section{Conditions for normality}

\begin{proposition}
    \label{thm:metric-space-normal}
    Pseudometric spaces are normal.
\end{proposition}

\begin{proof}
    Let $(S,\rho)$ be a pseudometric space, and let $A, B \subseteq S$ be disjoint closed subsets. For $a \in A$ let $r_a = \rho(a,B)/2 > 0$, and for $b \in B$ let $r_b = \rho(b,A)/2 > 0$. Let
    %
    \begin{equation*}
        U = \bigunion_{a \in A} B(a,r_a)
        \quad \text{and} \quad
        V = \bigunion_{b \in B} B(b,r_b).
    \end{equation*}
    %
    We claim that $U$ and $V$ are disjoint. Let $x \in U$ and $y \in V$. Then $x \in B(a,r_a)$ and $y \in B(b,r_b)$ for some $a \in A$ and $b \in B$. Then
    %
    \begin{equation*}
        \rho(a,b)
            \leq \rho(a,x) + \rho(x,y) + \rho(y,b)
            < \rho(x,y) + r_a + r_b,
    \end{equation*}
    %
    which implies that
    %
    \begin{equation*}
        0
            \leq \rho(a,b) - r_a - r_b
            < \rho(x,y),
    \end{equation*}
    %
    where the first inequality follows since
    %
    \begin{equation*}
        \rho(a,b)
            = \frac{\rho(a,b) + \rho(a,b)}{2}
            \geq \frac{\rho(a,B)}{2} + \frac{\rho(b,A)}{2}
            = r_a + r_b.
    \end{equation*}
\end{proof}



\begin{proposition}
    \label{thm:paracompact-Hausdorff-is-normal}
    Every paracompact Hausdorff space is normal, hence $T_4$.
\end{proposition}
% Hausdorff can be slightly weakened. Cool link: https://math.stackexchange.com/questions/1315614/r-1-paracompact-spaces-are-normal
% Also: https://en.wikipedia.org/wiki/Separation_axiom

\begin{proof}
    
\end{proof}


\begin{proposition}
    A regular Lindelöf space is normal.
\end{proposition}

\begin{proof}
    Let $X$ be a regular Lindelöf space, and let $A,B \subseteq X$ be disjoint closed subsets. By regularity, every $a \in A$ has a neighbourhood $U_a$ such that $\closure{U_a} \intersect B = \emptyset$. Similarly, every $b \in B$ has a neighbourhood $V_b$ separating it from $A$. Since $A$ and $B$ are themselves Lindelöf by \cref{thm:compact-Lindelof-closed-subset}, they are covered by countably many $U_a$ and $V_b$ respectively, say $A \subseteq \bigunion_{n \in \naturals} U_n$ and $B \subseteq \bigunion_{n \in \naturals} V_n$.

    Now define sequences of sets $S_n$ and $T_n$ by
    %
    \begin{equation*}
        S_n = U_n \setminus \closure{ \bigunion_{i < n} T_i}
        \quad \text{and} \quad
        T_n = V_n \setminus \closure{ \bigunion_{i \leq n} S_i}.
    \end{equation*}
    %
    (Notice the strict and non-strict inequalities.) Define the sets $S = \bigunion_{n \in \naturals} S_n$ and $T = \bigunion_{n \in \naturals} T_n$. Clearly $S$ is a neighbourhood of $A$ and $T$ of $B$. We claim that they are also disjoint: Let $x \in S_n$ for some $n \in \naturals$. Then $x \not\in T_m$ for $m < n$ by the definition of $S_n$, and $x \not\in T_m$ for $m \geq n$ by the definition of $T_m$. 
\end{proof}


\section{Urysohn's Lemma and related results}

If $X$ is a topological space and $A,B \subseteq X$ are closed sets, then a continuous function $f \colon X \to [0,1]$ with $f(A) = 0$ and $f(B) = 1$ is called a \emph{Urysohn function} for $A$ and $B$.

\begin{theorem}[Urysohn's Lemma]
    A topological space $X$ is normal if and only if there is a Urysohn function for every pair of closed subsets of $X$.
\end{theorem}

\begin{proof}
    First assume that $X$ is normal and that $A,B \subseteq X$ are closed. By normality there is an open set $U_{1/2}$ such that
    %
    \begin{equation*}
        A
            \subseteq U_{1/2}
            \subseteq \closure{U}_{\!1/2}
            \subseteq B^c.
    \end{equation*}
    %
    Then $A$ and $U_{1/2}^c$ are disjoint closed sets, and so are $\closure{U}_{\!1/2}$ and $B$. Hence there exist open sets $U_{1/4}$ and $U_{3/4}$ such that
    %
    \begin{equation*}
        A
            \subseteq U_{1/4}
            \subseteq \closure{U}_{\!1/4}
            \subseteq U_{1/2}
            \subseteq \closure{U}_{\!1/2}
            \subseteq U_{3/4}
            \subseteq \closure{U}_{\!3/4}
            \subseteq B^c.
    \end{equation*}
    %
    Let $\Delta$ be the set of all dyadic rational numbers in $(0,1)$. We may thus recursively define for every $r \in \Delta$ a set $U$ with the following properties:
    %
    \begin{enumerate} % Proof enumerate??
        \item $A \subseteq U_r$ and $\closure{U}_r \subseteq B^c$ for each $r \in \Delta$, and

        \item \label{enum:Urysohn-proof-closure-inside-open} $\closure{U}_r \subseteq U_s$ if $r < s$, for $r,s \in \Delta$.
    \end{enumerate}
    %
    We furthermore let $U_1 = X$. Then define a function $f \colon X \to [0,1]$ by $f(x) = \inf \set{r}{x \in U_r}$. Since $A \subseteq U_r \subseteq B^c$ for all $r \in \Delta$, we clearly have $f(A) = 0$ and $f(B) = 1$, and that $0 \leq f(x) \leq 1$ for all $x \in X$.

    It remains to be shown that $f$ is continuous. Let $\alpha \in \reals$ and $x \in X$, and notice that $f(x) < \alpha$ if and only if $x \in U_r$ for some $r < \alpha$, which is true just when $x \in \bigunion_{r < \alpha} U_r$. Hence,
    %
    \begin{equation*}
        f\preim((-\infty,\alpha))
            = \bigunion_{r < \alpha} U_r
    \end{equation*}
    %
    is open. Similarly $f(x) > \alpha$ if and only if $x \not\in U_r$ for some $r > \alpha$, which is equivalent to $x \not\in \closure{U}_s$ for some $s > \alpha$ by property \cref{enum:Urysohn-proof-closure-inside-open} above. This is the case if and only if $x \in \bigunion_{s > \alpha} (\closure{U}_s)^c$. It follows that
    %
    \begin{equation*}
        f\preim((\alpha,\infty))
            = \bigunion_{s > \alpha} (\closure{U}_s)^c
    \end{equation*}
    %
    is also open. Hence $f$ is continuous.

    Conversely, assume that $f$ is a Urysohn function for a pair of disjoint closed sets $A,B \subseteq X$. Then $f\preim([0,1/2))$ and $f\preim((1/2,1])$ are disjoint neighbourhoods of $A$ and $B$ respectively, so $X$ is normal.
\end{proof}


\begin{theorem}[The Tietze extension theorem]
    \label{thm:Tietze-extension}
    A topological space $X$ is normal if and only if any continuous function $f \colon A \to \reals$ on a closed set $A \subseteq X$ can be extended to a continuous function on all of $X$, i.e. there exists a continuous $F \colon X \to \reals$ such that $f = F|_A$.

    Furthermore, if $a,b \in \reals$ and $f(A) \subseteq [a,b]$ then $F$ can be chosen such that $F(X) \subseteq [a,b]$.
\end{theorem}

\begin{proof}
    
\end{proof}


\nocite{*}

\printbibliography


\end{document}
