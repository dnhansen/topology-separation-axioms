% Document setup
\documentclass[article, a4paper, 11pt, oneside]{memoir}
\usepackage[utf8]{inputenc}
\usepackage[T1]{fontenc}
\usepackage[UKenglish]{babel}

% Document info
\newcommand\doctitle{Separation Axioms}
\newcommand\docauthor{Danny Nygård Hansen}

% Formatting and layout
\usepackage[autostyle]{csquotes}
\usepackage[final]{microtype}
\usepackage{xcolor}
\frenchspacing
\usepackage{latex-sty/articlepagestyle}
\usepackage{latex-sty/articlesectionstyle}

% Fonts
\usepackage[largesmallcaps]{kpfonts}
\DeclareSymbolFontAlphabet{\mathrm}{operators} % https://tex.stackexchange.com/questions/40874/kpfonts-siunitx-and-math-alphabets
\linespread{1.06}
\let\mathfrak\undefined
\usepackage{eufrak}
\usepackage{inconsolata}
% \usepackage{amssymb}

% Hyperlinks
\usepackage{hyperref}
\definecolor{linkcolor}{HTML}{4f4fa3}
\hypersetup{%
	pdftitle=\doctitle,
	pdfauthor=\docauthor,
	colorlinks,
	linkcolor=linkcolor,
	citecolor=linkcolor,
	urlcolor=linkcolor,
	bookmarksnumbered=true
}

% Equation numbering
\numberwithin{equation}{chapter}

% Footnotes
\footmarkstyle{\textsuperscript{#1}\hspace{0.25em}}

% Mathematics
\usepackage{latex-sty/basicmathcommands}
\usepackage{latex-sty/framedtheorems}
\usepackage{latex-sty/topologycommands}
\usepackage{tikz-cd}
\usetikzlibrary{babel}

% Lists
\usepackage{enumitem}
\setenumerate[0]{label=\normalfont(\arabic*)}

% Bibliography
\usepackage[backend=biber, style=authoryear, maxcitenames=2, useprefix]{biblatex}
\addbibresource{references.bib}

% Title
\title{\doctitle}
\author{\docauthor}


% Section style -- add to section style .sty?
\setsubsecheadstyle{\normalfont\itshape}


\newcommand{\calU}{\mathcal{U}}
\newcommand{\calV}{\mathcal{V}}
\newcommand{\calW}{\mathcal{W}}
\newcommand{\calT}{\mathcal{T}}
\newcommand{\calN}{\mathcal{N}}
\newcommand{\calP}{\mathcal{P}}
\newcommand{\calB}{\mathcal{B}}
\newcommand{\calS}{\mathcal{S}}
\DeclarePairedDelimiter{\nhoodfilteraux}{(}{)}
% \newcommand{\nhoodfilter}[1]{\calN\nhoodfilteraux{#1}}
\newcommand{\nhoodfilter}[1]{\calN_{#1}}
\renewcommand{\powerset}[1]{\calP(#1)}

\renewcommand{\implies}{\Rightarrow}
\renewcommand{\iff}{\Leftrightarrow}

\renewcommand\interior[2][]{%
    \ifstrempty{#1}{%
        #2^{\circ}
    }{%
        \interiorop_{#1}#2
    }%
}


\begin{document}

\maketitle

\chapter{Introduction}

In these notes we give an overview of some of the most important separation axioms in point-set topology. For each axiom we consider the following:
%
\begin{enumerate}
    \item What the axiom \emph{means} intuitively. Of course we understand what it means to \textquote{separate} points or sets in a topological space and how the ability to do so is useful. But many of the axioms are equivalent to properties of a space that (seemingly) have nothing to do with separation, and whose importance are perhaps more intuitively clear.

    \item How the axiom behaves with respect to topological constructions like subspaces and products.
    
    \item Conditions under which a space satisfies a given axiom.

    \item Other properties of the axioms. Each axiom interacts with the surrounding theory in different ways, sometimes to produce spaces with even more structure.
\end{enumerate}
%
Our attention is primarily focused on both ends of the spectrum: The weakest axiom we consider, and the one we have most to say about, is $T_0$. We say less about the $T_1$ and Hausdorff axioms since these are well-known, and there is also rather little to say about regular spaces. More interesting are complete regularity and normality on which we spend a lot of time: normality because of its importance and particularly the Urysohn lemma, and complete regularity because of its connection with rings of continuous functions.


\section{Notation}

If $X$ and $Y$ are topological spaces, we denote the set of continuous maps $X \to Y$ by $C(X,Y)$. If $Y$ is a real/complex topological vector space, then $C(X,Y)$ is a real/complex vector space when addition and scalar multiplication are defined pointwise. In the case $Y = \reals$ we simply write $C(X)$, and this is an algebra over $\reals$. We also write $C_b(X)$ for the subalgebra of bounded functions.


\section{Overview of the axioms}

We here summarise the axioms by citing

\begin{itemize}
    \item $T_0$: The topology respects the underlying set structure by distinguishing between distinct points.
    
    \item $T_1$: Singletons are closed, or in other words, if $(x_\alpha)_{\alpha \in A}$ is a constant net with $x_\alpha = x$ for all $\alpha \in A$, then $x$ is the unique limit of $(x_\alpha)$.
    
    \item Hausdorff: All limits are unique (\cref{thm:Hausdorff-equivalent-properties}).
    
    \item Regularity: Every open neighbourhood contains a closed neighbourhood (\cref{thm:regular-equivalent-properties}).
    
    \item Complete regularity: Has enough bounded continuous functions to determine its topology (\cref{thm:completely-regular-weak-topology}).
    
    \item Normality: Continuous functions on closed sets can be extended to the entire space (\cref{thm:Tietze-extension}). Why should we restrict to \emph{closed} sets? Continuous functions defined on a non-closed set may not even extend to a continuous function on the \emph{closure} of that set: Take for instance the function $x \mapsto 1/x$ on $\reals \setminus \{0\}$.
\end{itemize}


\chapter{Preliminary definitions and results}


If $X$ is a topological space and $A \subseteq X$, then we say that a set $N \subseteq X$ is a \emph{neighbourhood} of $A$ if there is an open set $U$ in $X$ such that $A \subseteq U \subseteq N$. The family of neighbourhoods of a set $A$ is called the \emph{neighbourhood filter of $A$} and is denoted $\nhoodfilter{A}$. A \emph{neighbourhood basis at $A$} is a filter basis for $\nhoodfilter{A}$. If $A = \{x\}$ is a singleton we also write $\nhoodfilter{x}$ and call $N$ a neighbourhood of $x$.

If $X$ is a topological space, then we say that two subsets $A$ and $B$ are \emph{separated} if either $A$ has a neighbourhood disjoint from $B$, or vice versa. Separated sets are clearly disjoint. We say that $A$ and $B$ are \emph{separated by neighbourhoods} if there exist neighbourhoods of the two sets that are disjoint.

We begin by proving a series of results that do not pertain directly to the separability axioms, but that will be used in the sequel.


\section{Compactness}

\newcommand{\calA}{\mathcal{A}}

If $X$ is a topological space, we say that a collection $\calU$ of subsets of $X$ is a \emph{cover} of $X$ if $X = \bigunion_{U \in \calU} U$. A \emph{subcover} of a cover $\calU$ is a subcollection of $\calU$ that itself is a cover. An \emph{open cover} is a cover consisting of open sets.

If $\calU$ and $\calV$ are covers of $X$, we say that $\calU$ \emph{refines} $\calV$ and is a \emph{refinement} of $V$ if each $U \in \calU$ is contained in some $V \in \calV$. Notice that a subcover is in particular a refinement.

A collection $\calA$ of subsets of $X$ is called \emph{locally finite} if every point of $X$ has a neighbourhood that intersects finitely many elements in $\calA$. It is easy to show that if $\calA$ is locally finite, then $\closure{\calA} = \set{\closure{A}}{A \in \calA}$ is also locally finite: For if $x \in X$ and $U \subseteq X$ is an open neighbourhood of $x$ and $A \in \calA$ does not intersect $U$, then $A \subseteq U^c$, and so $\closure{A} \subseteq U^c$.

\begin{definition}
    Let $X$ be a topological space.
    %
    \begin{enumdef}
        \item $X$ is \emph{compact} if every open cover of $X$ has a finite subcover.
        
        \item $X$ is \emph{countably compact} if every countable open cover of $X$ has a finite subcover.

        \item $X$ is \emph{Lindelöf} if every open cover of $X$ has a countable subcover.

        \item $X$ is \emph{paracompact} if every open cover of $X$ has an open locally finite refinement.
    \end{enumdef}
\end{definition}
%
Clearly a space is compact if and only if it is both countably compact and Lindelöf, and a compact space is also paracompact. A subset $A \subseteq X$ is called \emph{precompact} in $X$ if its closure $\closure{A}$ in $X$ is compact.


\begin{proposition}
    \label{thm:compact-Lindelof-closed-subset}
    Compactness, countable compactness, Lindelöf and paracompactness are weakly hereditary.
\end{proposition}
%
This result is of course standard for compact spaces, but we include the proof to illustrate that it is identical to the proof for the other cases.

\begin{proof}
    Let $X$ be compact/countably compact/Lindelöf/paracompact, and let $A \subseteq X$ be closed. If $\calU$ is an open cover (countable if $X$ is countably compact) of $A$, then there is a (again countable) collection $\calV$ of sets open in $X$ such that $\calU = \set{V \intersect A}{V \in \calV}$. By adjoining $A^c$ to $\calV$ we obtain an open cover of $X$ (which is countable), so it has a finite/countable/locally finite subcover $\calV'$. Discarding $A^c$ and intersecting every other set in $\calV'$ with $A$ yields a finite/countable/locally finite subcover of $\calU$, proving the claim.
\end{proof}


\begin{proposition}
    The continuous image of a compact/countably compact/Lindelöf space is compact/countably compact/Lindelöf.
\end{proposition}

\begin{proof}
    Let $f \colon X \to Y$ be continuous, and assume that $X$ is compact/countably compact/Lindelöf. Let $\calU$ be an open cover of $f(X)$ (countable if $X$ is countably compact). Then $f\preim(\calU)$ is an open cover of $X$ (again countable), so it has a finite/countable subcover $f\preim(\calU')$. Notice that
    %
    \begin{equation*}
        f(X)
            = f \bigl( f\preim \bigl( \bigunion \calU' \bigr) \bigr)
            \subseteq \bigunion \calU',
    \end{equation*}
    %
    so $\calU'$ is a finite/countable subcover of $\calU$ as desired.
\end{proof}


\begin{remark}
    Note that the continuous image of a paracompact space need not be paracompact. Let $(X,\calT)$ be a space that is not paracompact, and let $\calT_{\mathrm{d}}$ be the discrete topology on $X$. Then $(X,\calT_{\mathrm{d}})$ is paracompact, but the identity map $\id_X \colon (X,\calT_{\mathrm{d}}) \to (X,\calT)$ is continuous.
\end{remark}


\begin{lemma}
    \label{thm:separating_from_compacts}
    Let $X$ be a topological space, and let $A \subseteq X$ be a set that can be separated from points that lie in a set $B \subseteq X$. Then $A$ can be separated from compact sets contained in $B$.
\end{lemma}
%
This is an example of the general principle that compact sets often act like points.

\begin{proof}
    Let $K \subseteq X$ be a compact set contained in $B$. Since $A$ can be separated from points in $B$, then for every $x \in K$ there are disjoint open sets $U_x$ and $V_x$ such that $x \in U_x$ and $A \subseteq V_x$. The collection $(U_x)_{x \in K}$ is an open cover of $K$, so there is a finite subcover $(U_{x_i})_{i=1}^n$. Let $U = \bigunion_{i=1}^n U_{x_i}$ and $V = \bigintersect_{i=1}^n V_{x_i}$. Then $U$ and $V$ are disjoint open sets containing $K$ and $A$ respectively.
\end{proof}


\section{Local compactness}

\begin{definition}[Local compactness]
    \label{def:local-compactness}
    A topological space is called
    %
    \begin{enumdef}
        \item \emph{weakly locally compact} if every point has a compact neighbourhood,
        \item \emph{weakly locally precompact} if every point has a precompact (open) neighbourhood,
        \item \emph{(strongly) locally compact} if every point has a neighbourhood basis of compact sets, and
        \item \emph{(strongly) locally precompact} if every point has a neighbourhood basis of precompact (open) sets.
    \end{enumdef}
\end{definition}
%
Clearly a point has a precompact neighbourhood if and only if it has a precompact \emph{open} neighbourhood. A locally compact Hausdorff space is also called an \emph{LCH space}.

\begin{proposition}
    \label{thm:LCH-equivalent-condition}
    Let $X$ be a topological space. If $X$ has any of the properties in \cref{def:local-compactness}, then $X$ is weakly locally compact. If $X$ is Hausdorff, then all properties in \cref{def:local-compactness} are equivalent.
\end{proposition}

\begin{proof}
    The first claim is obvious. For the second claim, assume that $X$ is Hausdorff and weakly locally compact, and let $x \in X$. Then $x$ has a compact neighbourhood $K$, i.e., there is an open set $U$ with $x \in U \subseteq K$. Since $K$ is closed we have $\closure{U} \subseteq K$, so $U$ is precompact.

    Next let $U$ be a precompact open neighbourhood of $x$. Then $\boundary U$ is compact, so $x$ and $\boundary U$ are separated by \cref{thm:Hausdorff-separate-compacts}, i.e., there exist disjoint open sets $V,W$ such that $x \in V$ and $\boundary U \subseteq W$. Replacing $V$ with $V \intersect U$ we may assume that $V \subseteq U \setminus W$. Hence $\closure{V} \subseteq \closure{U} \setminus W \subseteq U$, so $X$ is (strongly) locally compact.
\end{proof}


\begin{lemma}
    \label{lem:local-compactness-hereditary}
    All properties in \cref{def:local-compactness} are weakly hereditary. Open subsets of strongly locally compact spaces are also strongly locally compact.
\end{lemma}

\begin{proof}
    The first claim is obvious, since the intersection of a compact set and a closed set is compact. The second claim is also obvious.
\end{proof}


\begin{lemma}
    \label{thm:LCH-compact-set-has-compact-nhood}
    If $X$ is weakly locally compact and $K \subseteq X$ is compact, then $K$ has a compact neighbourhood. In particular, if $X$ is strongly locally compact and $K \subseteq U \subseteq X$ with $U$ open, then $K$ has a compact neighbourhood contained in $U$.
\end{lemma}

\begin{proof}
    For the first claim, note that each point of $K$ has a compact neighbourhood, hence is covered by finitely many such neighbourhoods. Their union is a compact neighbourhood of $K$. The second claim then follows from \cref{lem:local-compactness-hereditary}.
\end{proof}


\section{Other types of compactness}

\begin{definition}
    \label{def:hemicompact-et-al}
    Let $X$ be a topological space. Then $X$ is called
    %
    \begin{enumdef}
        \item \emph{exhaustible by compact sets} if there is a sequence $(K_n)_{n \in \naturals}$ of subsets of $X$, called an \emph{exhaustion by compact sets}, such that $K_n \subseteq \interior{K_{n+1}}$ and $X = \bigunion_{n \in \naturals} K_n$.
        
        \item \emph{hemicompact} if there is a sequence $(K_n)_{n \in \naturals}$ of subsets of $X$, called an \emph{admissible sequence}, such that for every compact $K \subseteq X$ there is an $n \in \naturals$ with $K \subseteq K_n$.

        \item \emph{$\sigma$-compact} if there is a sequence $(K_n)_{n \in \naturals}$ of subsets of $X$ such that $X = \bigunion_{n \in \naturals} K_n$.
    \end{enumdef}
\end{definition}


\begin{proposition}
    In the notation of \cref{def:hemicompact-et-al}, we have the following implications: (i) $\Rightarrow$ (ii) $\Rightarrow$ (iii) $\Rightarrow$ Lindelöf. If $X$ is weakly locally compact, then these properties are equivalent.
\end{proposition}

\begin{proof}
    If $(K_n)_{n \in \naturals}$ is an exhaustion of $X$ and $K \subseteq X$ is compact, then $K$ is covered by finitely many $K_n$. Choosing the largest of these $K_n$ shows that $X$ is hemicompact.

    If $X$ is hemicompact, then an admissible sequence covers $X$ since any singleton is compact. If $X$ is $\sigma$-compact, $(K_n)_{n \in \naturals}$ is a sequence of compact sets that cover $X$, and $\calU$ is a open covering of $X$, then each $K_n$ is covered by finitely many sets from $\calU$, and so $X$ is covered by countably many sets from $\calU$.

    Conversely assume that $X$ is weakly locally compact. If $X$ is Lindelöf, let $K_x$ be a compact neighbourhood of $x \in X$. Then $X$ is covered by the interiors of the $K_x$, hence by countably many such sets. But then $X$ is $\sigma$-compact.

    If $X$ is instead $\sigma$-compact, let $(K_n)_{n \in \naturals}$ be a covering of $X$ by compact sets. Let $C_1 = K_1$ and assume that $C_1, \ldots, C_n$ have been defined with $C_i \subseteq \interior{C_{i+1}}$. Then $C_n \union K_n$ is compact and has a compact neighbourhood $C_{n+1}$ by \cref{thm:LCH-compact-set-has-compact-nhood}. The sequence $(C_n)_{n \in \naturals}$ is thus an exhaustion of $X$ by compact sets.
\end{proof}


\begin{proposition}
    If $X$ is hemicompact and first countable, then $X$ is weakly locally compact.
\end{proposition}

\begin{proof}
    Let $(K_n)_{n \in \naturals}$ be an admissible sequence in $X$ with $K_n \subseteq K_{n+1}$, and assume that there is a point $x \in X$ with no compact neighbourhood. If $(U_n)_{n \in \naturals}$ is a neighbourhood basis at $x$ with $U_n \subseteq U_{n+1}$, then since $K_n$ is not a neighbourhood of $x$ there is a point $x_n \in U_n \setminus K_n$. The set $\set{x_n}{n \in \naturals} \union \{x\}$ is then compact but is not contained in any $K_n$.
\end{proof}



\section{Metric spaces}

\begin{lemma}
    \label{thm:distance-to-set}
    Let $(S,\rho)$ be a pseudometric space, and let $A \subseteq S$ be nonempty. Define a map $\rho(\,\cdot\,, A) \colon S \to [0, \infty)$ by
    %
    \begin{equation*}
        \rho(x, A) = \inf_{a \in A} \rho(x,a).
    \end{equation*}
    %
    This map has the following properties:
    %
    \begin{enumlem}
        \item \label{enum:distance-to-set-closure} $\rho(x,A) = 0$ if and only if $x \in \closure{A}$.

        \item \label{enum:distance-to-set-triangle-inequality} If $y \in S$, then $\rho(x,A) \leq \rho(x,y) + \rho(y,A)$.

        \item \label{enum:distance-to-set-continuous} $\rho(\,\cdot\,, A)$ is continuous.
    \end{enumlem}
\end{lemma}

\begin{proof}
    First we prove \subcref{enum:distance-to-set-closure}. Notice that $\rho(x,A) = 0$ if and only if for any $r > 0$ there is an $a \in A$ such that $\rho(x,a) < r$. But this is true precisely when any ball $B(x,r)$ intersects $A$, i.e. when $x \in \closure{A}$.

    For any $a \in A$ we have
    %
    \begin{equation*}
        \rho(x,A)
            \leq \rho(x,a)
            \leq \rho(x,y) + \rho(y,a),
    \end{equation*}
    %
    and since this is true for any $a \in A$, \subcref{enum:distance-to-set-triangle-inequality} follows.

    Finally, \subcref{enum:distance-to-set-continuous} follows immediately from \subcref{enum:distance-to-set-triangle-inequality}, since
    %
    \begin{equation*}
        \rho(x,A) - \rho(y,A)
            \leq \rho(x,y)
    \end{equation*}
    %
    for all $x,y \in S$.
\end{proof}



\chapter[The T0 axiom][The $T_0$ axiom]{The $T_0$ axiom}

\section[Definition and the T0-identification][Definition and the $T_0$-identification]{Definition and the $T_0$-identification}

\begin{definition}
    A topological space $X$ satisfies the $T_0$ axiom and is called a \emph{$T_0$-space} or said to be \emph{Kolmogorov} if, for every pair of distinct points $x,y \in X$, either $x$ has a neighbourhood that does not contain $y$, or vice versa.
\end{definition}

We begin by giving an alternative characterisation of $T_0$-spaces: Let $X$ be a topological space. We define an ordering on $X$ called the \emph{specialisation preorder} by letting $x \leq y$ if $\nhoodfilter{x} \subseteq \nhoodfilter{y}$, or equivalently if $x \in \closure{\{y\}}$, for $x,y \in X$. It is clear that $\leq$ is in fact a preorder, and so it determines an equivalence relation $\equiv$; that is, $x \equiv y$ if and only if $x \leq y$ and $y \leq x$.

If $x \equiv y$, then we say that $x$ and $y$ are \emph{topologically indistinguishable} since then $x$ and $y$ have the same neighbourhoods, i.e., $\nhoodfilter{x} = \nhoodfilter{y}$.

It is clear that $X$ is $T_0$ if and only if the relation $\equiv$ is trivial, that is if $x$ and $y$ are topologically indistinguishable precisely when $x = y$. The quotient space $X/{\equiv}$ is called the \emph{$T_0$-identification} or the \emph{Kolmogorov quotient} of $X$, and it is indeed $T_0$:

\begin{theorem}[The $T_0$-identification]
    Let $X$ be a topological space, and let $q \colon X \to X/{\equiv}$ be the quotient map onto the $T_0$-identification of $X$. Then
    %
    \begin{enumthm}
        \item \label{enum:T0-identification-saturated} open and closed sets are saturated,
        \item \label{enum:T0-quotient-map-open-closed} $q$ is an open and closed map,
        \item $X/{\equiv}$ is $T_0$, and
        \item \label{enum:T0-identification-is-smallest-equiv} if $\sim$ is an equivalence relation on $X$ such that $X/{\sim}$ is $T_0$, then ${\equiv} \subseteq {\sim}$.\footnotemark
    \end{enumthm}
\end{theorem}
\footnotetext{Both $\equiv$ and $\sim$ are subsets of $X \times X$, so this inclusion means that, for all $x,y \in X$, if $x \equiv y$ then $x \sim y$.}
%
Part \subcref{enum:T0-identification-is-smallest-equiv} expresses the fact that $\equiv$ is the most conservative equivalence relation on $X$ that makes the corresponding quotient a $T_0$-space.

\begin{proof}
    We first show that all open sets are saturated. Let $U$ be an open set of $X$, and let $x \in U$. If $x \equiv x'$, then $U$ is also a neighbourhood of $x'$, so $x' \in U$; in other words, $U$ is a union of fibres. Hence $U$ is saturated with respect to $q$. Complements of saturated sets are also unions of fibres, hence saturated, so closed sets are also saturated. Since $q$ is a quotient map, it follows that it takes saturated open (closed) subsets of $X$ to open (closed) subsets of $X/{\equiv}$,\footnote{If $U \subseteq X$ is saturated, then $q(U)$ is open in $Y$ if and only if $q\preim(q(U)) = U$ is open in $X$. See also \textcite[Proposition~3.60]{leetopological}.} and hence it is both open and closed.

    Now we show that $X/{\equiv}$ is $T_0$. Assume that $x \not\equiv y$. Without loss of generality we may assume the existence of an element $U \in \nhoodfilter{x} \setminus \nhoodfilter{y}$, and that $U$ is open. Since $q$ is open, $q(U)$ is a neighbourhood of $q(x)$ in $X/{\equiv}$. And $U$ is saturated so $U = q\inv(q(U))$, and because $y \not\in U$ it follows that $q(y) \not\in q(U)$. Hence $q(U)$ is a neighbourhood of $q(x)$ that is not a neighbourhood of $q(y)$, and thus $X/{\equiv}$ is $T_0$.

    Finally, let $X/{\sim}$ be $T_0$, and let $p \colon X \to X/{\sim}$ be the quotient map. If $x \not\sim y$ then $p(x) \neq p(y)$, so without loss of generality we may choose an open set $U \subseteq X/{\sim}$ with $p(x) \in U$ and $p(y) \not\in U$. Then $x \in p\preim(U)$ and $y \not\in p\preim(U)$, i.e. $p\preim(U)$ is a neighbourhood of $x$ that is not a neighbourhood of $y$, so $x \not\equiv y$.
\end{proof}


\begin{corollary}
    Let $(X,\calT)$ be a topological space, and let $q \colon X \to X/{\equiv}$ be the quotient map onto the $T_0$-identification of $X$. Denote the topology on $X/{\equiv}$ by $\calT_\equiv$. Then $q$ induces a bijection $q_* \colon \calT \to \calT_\equiv$ given by $q_*(U) = q(U)$ whose inverse $q^*$ is given by $q^*(V) = q\preim(V)$.
\end{corollary}

\begin{proof}
    Since $q$ is surjective we have $q(q\preim(V)) = V$ for all $V \in \calT_\equiv$, and by \cref{enum:T0-identification-saturated} every $U \in \calT$ is saturated, so $q\preim(q(U)) = U$.
\end{proof}
%
This corollary implies that most topological properties are preserved in the $T_0$-identification. Taking the $T_0$-identification of a space that is already $T_0$ leaves the space unchanged but, preempting terminology we will introduce later, the $T_0$-identification of a regular space is regular, and the same is true for completely regular, normal and paracompact spaces. The proofs are trivial.

One might also expect that continuous functions on a space are unchanged in the $T_0$-identification, and this is in fact the case:

\begin{proposition}
    \label{thm:T0-identification-continuous-functions}
    Let $X$ be a topological space, and let $q \colon X \to X/{\equiv}$ be the quotient map onto its $T_0$-identification. For every $T_0$-space $Y$ the pullback map
    %
    \begin{align*}
        q^* \colon C(X/{\equiv},Y) &\to C(X,Y), \\
        f &\mapsto f \circ q,
    \end{align*}
    %
    is a bijection. If $Y$ is a $T_0$ topological vector space\footnotemark, then $q^*$ is a linear isomorphism, and it is an algebra isomorphism in the case $Y = \reals$. % TODO: What about more abstract R-algebras?
\end{proposition}

\begin{proof}\footnotetext{One can show that a topological group is $T_1$ if it is $T_0$, and furthermore is always regular, so a $T_0$ topological group is $T_3$.}
    Let $q \colon X \to X/{\equiv}$ be the quotient map, and let $g \in C(X,Y)$. We claim that if $x \equiv y$ in $X$, then $g(x) = g(y)$. For if $g(x) \neq g(y)$, then since $Y$ is $T_0$ we can, without loss of generality, choose a neighbourhood $U$ of $g(x)$ not containing $g(y)$. Then $g\preim(U)$ is a neighbourhood of $x$ not containing $y$, so $x \not\equiv y$.

    Thus every $g \in C(X,Y)$ descends to a map $\tilde{g} \in C(X/{\equiv},Y)$ with $g = \tilde{g} \circ q$, showing surjectivity, and $\tilde{g}$ is unique, showing injectivity. Hence $q^*$ is a bijection.

    Now let $Y$ be a $T_0$ topological vector space over $\reals$. Let $f,g \in C(X/{\equiv},Y)$ and $\alpha \in \reals$. Then
    %
    \begin{equation*}
        q^*(\alpha f + g)
            = (\alpha f + g) \circ q
            = \alpha (f \circ q) + (g \circ q)
            = \alpha q^*(f) + q^*(g).
    \end{equation*}
    %
    Hence $q^*$ is linear and thus a linear isomorphism.

    Finally, if $Y = \reals$ then $q^*$ respects multiplication by a similar argument to the above and is thus an algebra isomorphism.
\end{proof}


\begin{proposition}
    \label{prop:specialisation-preorder-initial-topology}
    Let $X$ carry the initial topology induced by maps $f_\alpha \colon X \to X_\alpha$ for $\alpha \in A$. For all $x,y \in X$ we have $x \leq y$ (resp. $x \equiv y$) if and only if $f_\alpha(x) \leq f_\alpha(y)$ (resp. $f_\alpha(x) \equiv f_\alpha(y)$) for all $\alpha \in A$.
\end{proposition}

\begin{proof}
    Let $x,y \in X$, and assume that $x \leq y$. If $U \in \nhoodfilter{f_\alpha(x)}$, then $f_\alpha\preim(U) \in \nhoodfilter{x} \subseteq \nhoodfilter{y}$ by continuity. That is, $y \in f_\alpha\preim(U)$, so $f_\alpha(y) \in U$ implying that $U \in \nhoodfilter{f_\alpha(y)}$. Hence $f_\alpha(x) \leq f_\alpha(y)$ as claimed.

    Conversely, assume that $f_\alpha(x) \leq f_\alpha(y)$ for all $\alpha \in A$. Notice that it suffices to show that $\calB_x \subseteq \nhoodfilter{y}$ for some neighbourhood basis $\calB_x$ at $x$. We construct $\calB_x$ as follows: If $\calT_\alpha$ is the topology on $X_\alpha$ and $\calT$ the topology on $X$, then $\calS = \bigunion_{\alpha \in A} f_\alpha\preim(\calT_\alpha)$ is a subbasis for $\calT$, and the set $\calB$ of finite intersections of elements in $\calS$ is a basis for $\calT$. We then let $\calB_x = \calB \intersect \nhoodfilter{x}$.

    An arbitrary element in $\calB_x$ is thus on the form $U = \bigintersect_{i=1}^n f_{\alpha_i}\preim(U_i)$, with $U_i \in \calT_{\alpha_i}$. It follows that $f_{\alpha_i}(x) \in U_i$ so that $U_i \in \nhoodfilter{f_{\alpha_i}(x)} \subseteq \nhoodfilter{f_{\alpha_i}(y)}$, implying that that $f_{\alpha_i}(y) \in U_i$. Hence $y \in f_{\alpha_i}\preim(U_i)$, so $y \in U$ and $U \in \nhoodfilter{y}$ as desired.
\end{proof}


We explore the $T_0$-identification in the context of metric spaces: Let $(S,\rho)$ be a pseudometric space, and define a relation $\sim$ on $S$ by $x \sim y$ if and only if $\rho(x,y) = 0$. This is clearly an equivalence relation. Let $\tilde{S} = S/{\sim}$ and define a map $\tilde{\rho} \colon \tilde{S} \prod \tilde{S} \to [0,\infty)$ by $\tilde{\rho}([x],[y]) = \rho(x,y)$. This is well-defined, since if $x \sim x'$ and $y \sim y'$ then
%
\begin{equation*}
    \rho(x,y)
        \leq \rho(x,x') + \rho(x',y') + \rho(y',y)
        = \rho(x',y').
\end{equation*}
%
It is obvious that $\tilde{\rho}$ is then a metric on $\tilde{S}$, and we call $(\tilde{S}, \tilde{\rho})$ the \emph{metric identification} of $(S,\rho)$.

\begin{proposition}
    Let $(S,\rho)$ be a pseudometric space. Then the relation $\sim$ defined above and the topological indistinguishability relation $\equiv$ coincide.
\end{proposition}

\begin{proof}
    It suffices to show that $x \sim y$ if and only if $\nhoodfilter{x} = \nhoodfilter{y}$ for all $x,y \in S$. Assume that $x \sim y$ and let $U \in \nhoodfilter{x}$. Then there is an $r > 0$ such that $B(x,r) \subseteq U$. But we clearly have $y \in B(x,r)$, so $U \in \nhoodfilter{y}$. This shows that $\nhoodfilter{x} \subseteq \nhoodfilter{y}$, and the opposite inclusion follows by symmetry.

    Conversely, assume that $x \not\sim y$. Then $0 < r < \rho(x,y)$ for some $r$, and $B(x,r)$ is a neighbourhood of $x$ but not of $y$.
\end{proof}


\section[Operations on T0-spaces][Operations on $T_0$-spaces]{Operations on $T_0$-spaces}


\begin{proposition}
    \label{thm:T0-initial-topology}
    Let $X$ carry the initial topology induced by maps $f_\alpha \colon X \to X_\alpha$ for $\alpha \in A$. Assume that each $X_\alpha$ is $T_0$, and that the $f_\alpha$ separate points in $X$. Then $X$ is also $T_0$.

    In particular, any subspace of a $T_0$-space is $T_0$, and a product of $T_0$-spaces is $T_0$. Furthermore, if a nonempty product space is $T_0$, then every factor is $T_0$. % TODO: Are there any interesting generalisations of this?
\end{proposition}
%
We give two proofs of this result, one using the characterisation of the specialisation preorder given in \cref{prop:specialisation-preorder-initial-topology}, and another more direct proof.

\begin{proofof}[Proof 1]
    Let $x,y \in X$ with $x \neq y$. Since the $f_\alpha$ separate points in $X$, there is a $\beta \in A$ such that $x_\beta \neq y_\beta$. Since $X_\beta$ is $T_0$, the specialisation preorder on $X_\beta$ is just equality, so $x_\beta \not\equiv y_\beta$. But then $x \not\equiv y$ by \cref{prop:specialisation-preorder-initial-topology}.

    The final claim follows directly from the same proposition, since if the product space is nonempty, then the projections are surjective.
\end{proofof}

\begin{proofof}[Proof 2]
    Let $x,y \in X$ with $x \neq y$. Since the $f_\alpha$ separate points in $X$, there is a $\beta \in A$ such that $x_\beta \neq y_\beta$. Without loss of generality pick a neighbourhood $U$ of $x_\beta$ in $X_\beta$ that does not contain $y_\beta$. Then $f_\beta\preim(U)$ is a neighbourhood of $x$ that does not contain $y$.

    For the final claim, assume that $X = \bigprod_{\alpha \in A} X_\alpha$ is a nonempty $T_0$ product space, and let $\beta \in A$. Pick a point $y \in X$, and let
    %
    \begin{equation*}
        Y
            = X_\beta \prod \bigprod_{\alpha \neq \beta} \{y_\alpha\}
            = \set{x \in X}{x_\alpha = y_\alpha \text{ for } \alpha \neq \beta}.
    \end{equation*}
    %
    Then $Y$ is $T_0$ since it is a subspace of $X$, so $X_\beta \cong Y$ is also $T_0$.
\end{proofof}
%
Notice that we could also have proved the first result above by proving it separately for subspaces and products, and then using the fact that $X$ could be embedded in the product $\bigprod_{\alpha \in A} X_\alpha$ since the $f_\alpha$ separate points\footnote{See e.g. \href{https://github.com/dnhansen/topology-measure-theory-notes}{my notes on measure theory and topology}, or \textcite[Theorem~8.12]{willard}.}. In fact, we may use this product embedding to characterise the $T_0$-spaces as follows:

First recall that the \emph{Sierpi\'nski space} is the space $S$ with the underlying set $\{0,1\}$ and the topology $\{\emptyset, \{1\}, S\}$, so that the specialisation preorder on $S$ is just the usual ordering on $\{0,1\} \subseteq \naturals$. If $X$ is any set, then the set of functions $S^X$ is just the indicator functions on subsets of $X$, so that $S^X \cong \powerset{X}$. If $(X,\calT)$ is a topological space, then an indicator function $\indicator{U} \colon X \to S$ is continuous just when $U = \indicator{U}\preim(1)$ is open. It follows that there is a bijection betwen $\calT$ and the set $C(X,S)$ of continuous maps $X \to S$.

Furthermore, $X$ has the initial topology induced by $C(X,S)$, since removing an open set $U$ from $\calT$ would make the (continuous) indicator function $\indicator{U}$ discontinuous. Also notice that $C(X,S)$ separates points in $X$ just when $X$ is $T_0$. Hence if $X$ is $T_0$, then $X$ can be embedded into the product $S^\calT$ by the map $f \colon X \to S^\calT$ with the property that $\pi_U \circ f = \indicator{U}$. Since subspaces and products of $T_0$-spaces are $T_0$, the converse also holds.


\begin{proposition}
    \label{thm:T0-disjoint-union}
    Let $(X_\alpha)_{\alpha \in A}$ be a collection of $T_0$-spaces. Then the disjoint union $X = \coprod_{\alpha \in A} X_\alpha$ is also $T_0$. % TODO: Iff, right? Because subspaces?
\end{proposition}
%
This proposition in fact holds for all final topologies on a set $X$ with the property that the coinducing maps $(f_\alpha)_{\alpha \in A}$ are injective, and the property that the images $f_\alpha(X_\alpha)$ form a partition of $X$. Under these assumptions the $f_\alpha$ cover $X$ and are both open and closed (the latter of which we shall not use below). I am not aware of any other interesting final topologies with these properties.

\begin{proof}
    First recall that the canonical injections $\iota_\alpha$ are open, and that their images form a partition of $X$. For $x,y \in X$ with $x \neq y$, if $x \in X_\alpha$ and $y \in X_\beta$ for $\alpha \neq \beta$ then e.g. $X_\alpha$ is a neighbourhood of $x$ not containing $y$. If instead $\alpha = \beta$, then since $X_\alpha$ is $T_0$ the point $x$ has a neighbourhood not containing $y$.
\end{proof}


% \begin{proposition}
%     \label{thm:T0-initial-topology}
%     \begin{enumprop}
%         \item \label{enum:T0-subspace} Any subspace of a $T_0$-space is $T_0$.
%         \item A nonempty product space is $T_0$ if and only if every factor is $T_0$.
%     \end{enumprop}
% \end{proposition}

% \begin{proof}
%     Let $X$ be a $T_0$-space and $A \subseteq X$. If $x,y \in A$ and $x \neq y$, then without loss of generality we may choose a neighbourhood $U$ of $x$ in $X$ that does not contain $Y$. But then $U \intersect A$ is a neighbourhood of $x$ in $A$ that doesn't contain $y$, so $A$ is $T_0$.

%     Now let $(X_\alpha)_{\alpha \in A}$ be a collection of topological spaces, and assume that the product $X = \bigprod_{\alpha \in A} X_\alpha$ is nonempty. Assume that all $X_\alpha$ are $T_0$, and let $x,y \in X$ be distinct points. Then there is a $\beta \in A$ such that $x_\beta \neq y_\beta$. Without loss of generality we may pick a neighbourhood $U$ of $x_\beta$ in $X_\beta$ that does not contain $y_\beta$. Then $\pi_\beta\preim(U)$ is a neighbourhood of $x$ that does not contain $y$.

%     Conversely, assume that the product $X$ is $T_0$ and let $\beta \in A$. Pick a point $y \in X$, and let
%     %
%     \begin{equation*}
%         Y
%             = \set{x \in X}{x_\alpha = y_\alpha \text{ for } \alpha \neq \beta}.
%     \end{equation*}
%     %
%     Then $Y$ is $T_0$ by \subcref{enum:T0-subspace}, so $X_\beta \cong Y$ is also $T_0$.
% \end{proof}



\chapter[The R0 and T1 axioms][The $R_0$ and $T_1$ axioms]{The $R_0$ and $T_1$ axioms}

\section{Definition and equivalent properties}

\begin{definition}
    A topological space $X$ satisfies the $R_0$ axiom and is called an \emph{$R_0$-space} or is said to be \emph{symmetric} if points $x,y \in X$ are separated whenever $x \not\equiv y$.
    
    Furthermore, $X$ satisfies the $T_1$ axiom and is called a \emph{$T_1$-space} or is said to be \emph{Fréchet} if $X$ is both $R_0$ and $T_0$, i.e., if any two distinct points in $X$ are separated.
\end{definition}
%
In any topological space we have the implications
%
\begin{equation*}
    \text{separated}
        \quad \implies \quad
        \text{topologically distinguishable}
        \quad \implies \quad
        \text{distinct}.
\end{equation*}
%
The first implication can be reversed just when the space is $R_0$, and the second just when the space is $T_0$. The composite arrow can thus be reversed when the space is $T_1$.

We begin by giving some properties of topological spaces that are equivalent to the two axioms:

\begin{proposition}
    The following are equivalent for a topological space $X$:
    %
    \begin{enumprop}
        \item \label{enum:T1-space} $X$ is $T_1$,
        \item \label{enum:T1-singletons-closed} each singleton of $X$ is closed, and
        \item \label{enum:T1-intersection-of-open-sets} each subset of $X$ is the intersection of all open sets containing it.
    \end{enumprop}
\end{proposition}

\begin{proof}
\begin{proofsec}
    \item[\subcref{enum:T1-space} $\implies$ \subcref{enum:T1-singletons-closed}]
    If $X$ is $T_1$ and $x \in X$, then every point $y \in X \setminus \{x\}$ has a neighbourhood disjoint from $\{x\}$ so $X \setminus \{x\}$ is open.

    \item[\subcref{enum:T1-singletons-closed} $\implies$ \subcref{enum:T1-intersection-of-open-sets}]
    If $A \subseteq X$, then
    %
    \begin{equation*}
        A   = X \setminus \bigunion_{x \not\in A} \{x\}
            = \bigintersect_{x \not\in A} X \setminus \{x\},
    \end{equation*}
    %
    so $A$ is an intersection of open sets.

    \item[\subcref{enum:T1-intersection-of-open-sets} $\implies$ \subcref{enum:T1-space}]
    If $x,y \in X$ with $x \neq y$, then there is an open subset containing $x$ and not $y$, and vice versa.
\end{proofsec}
\end{proof}


\begin{corollary}
    \label{cor:T1-quotient}
    A quotient space $X/{\sim}$ is $T_1$ if and only if every $\sim$-equivalence class is closed in $X$.
\end{corollary}

\begin{proof}
    Fibres of the quotient map are precisely the equivalence classes, so by the definition of the quotient topology, singletons of $X/{\sim}$ are closed if and only if the corresponding equivalence class is closed as a subset of $X$.
\end{proof}



\begin{proposition}
    The following are equivalent for a topological space $X$:
    %
    \begin{enumprop}
        \item \label{enum:R0-space} $X$ is $R_0$,
        \item \label{enum:R0-specialisation-preorder-symmetric} the specialisation preorder $\leq$ is symmetric (i.e., ${\leq} = {\equiv}$),
        \item \label{enum:R0-closure-of-point} $[x]_\equiv = \closure{\{x\}}$ for all $x \in X$,
        \item \label{enum:R0-equivalence-class-closed} $[x]_\equiv$ is closed for all $x \in X$,
        \item \label{enum:R0-partition} the sets $\closure{\{x\}}$ for $x \in X$ form a partition of $X$, and
        \item \label{enum:R0-quotient-is-T1} $X/{\equiv}$ is $T_1$.
    \end{enumprop}
\end{proposition}

\begin{proof}
\begin{proofsec}
    \item[\subcref{enum:R0-space} $\implies$ \subcref{enum:R0-specialisation-preorder-symmetric}]
    Assume that $x \leq y$. Then $x$ and $y$ are not separated, and so $x \equiv y$.

    \item[\subcref{enum:R0-specialisation-preorder-symmetric} $\implies$ \subcref{enum:R0-closure-of-point}]
    The inclusion \enquote{$\subseteq$} always holds, so assume that $y \in \closure{\{x\}}$. This means that $y \leq x$, so we also have $x \leq y$, i.e., $x \equiv y$.

    \item[\subcref{enum:R0-closure-of-point} $\iff$ \subcref{enum:R0-closure-of-point}]
    Since $\{x\} \subseteq [x]_\equiv \subseteq \closure{\{x\}}$, this is obvious.
    
    \item[\subcref{enum:R0-closure-of-point} $\implies$ \subcref{enum:R0-partition}]
    The $\equiv$-equivalence classes are a partition of $X$.
    
    \item[\subcref{enum:R0-partition} $\implies$ \subcref{enum:R0-space}]
    Assume that $x \not\equiv y$, and assume without loss of generality that $x \not\leq y$, i.e., that $x \not\in \closure{\{y\}}$. Then $\closure{\{x\}}$ and $\closure{\{y\}}$ are disjoint, and so $y \not\in \closure{\{x\}}$. Hence $x$ and $y$ are separated.

    \item[\subcref{enum:R0-equivalence-class-closed} $\iff$ \subcref{enum:R0-quotient-is-T1}]
    This follows from \cref{cor:T1-quotient}.
\end{proofsec}
\end{proof}



\section[Operations on R0- and T1-spaces][Operations on $R_0$- and $T_1$-spaces]{Operations on $R_0$- and $T_1$-spaces}

\begin{proposition}
    \label{prop:R0-initial-topology}
    Let $X$ carry the initial topology induced by maps $f_\alpha \colon X \to X_\alpha$ for $\alpha \in A$. If each $X_\alpha$ is $R_0$, then so is $X$. If $X$ is $R_0$ and $f_\alpha$ is surjective, then $X_\alpha$ is also $R_0$.
\end{proposition}

\begin{proof}
    Follows directly from \cref{prop:specialisation-preorder-initial-topology}.
\end{proof}


\begin{proposition}
    \label{thm:T1-initial-topology}
    Let $X$ carry the initial topology induced by maps $f_\alpha \colon X \to X_\alpha$ for $\alpha \in A$. Assume that each $X_\alpha$ is $T_1$, and that the $f_\alpha$ separate points in $X$. Then $X$ is also $T_1$.

    In particular, any subspace of a $T_1$-space is $T_1$, and a product of $T_1$-spaces is $T_1$. Furthermore, if a nonempty product space is $T_1$, then every factor is $T_1$.
\end{proposition}

\begin{proof}
    This follows from \cref{prop:R0-initial-topology} and \cref{thm:T0-initial-topology}.
\end{proof}


\begin{proposition}
    Let $(X_\alpha)_{\alpha \in A}$ be a collection of $T_1$-spaces. Then the disjoint union $X = \coprod_{\alpha \in A} X_\alpha$ is also $T_1$.
\end{proposition}

\begin{proof}
    Similar to the proof of \cref{thm:T0-disjoint-union}.
\end{proof}


\section[Conditions for the T1 axiom]{Conditions for the $T_1$ axiom}

\begin{proposition}
    The closed image\footnotemark{} of a $T_1$-space is $T_1$.
\end{proposition}

\begin{proof}\footnotetext{By \textquote{closed image} we mean the image of a closed (not necessarily continuous) map.}%
    Let $f \colon X \to Y$ be a closed map from a $T_1$-space $X$ to a topological space $Y$, and let $y \in f(X)$. Then there is some $x \in X$ with $f(x) = y$, and since $\{x\}$ is closed in $X$ and $f$ is closed, $\{y\}$ is closed in $Y$ and hence closed in $f(X)$.
\end{proof}



\chapter{Hausdorff spaces}

\section{Definition and equivalent properties}

\begin{definition}
    A topological space $X$ satisfies the $T_2$ axiom and is called a \emph{$T_2$-space} or \emph{Hausdorff space} if, for every pair of distinct points $x,y \in X$, $x$ has a neighbourhood $U$ and $y$ a neighbourhood $V$ with $U \intersect V = \emptyset$.
\end{definition}
%
Again, a $T_2$-space is clearly $T_1$. We give a series of conditions that are equivalent to the $T_2$ axiom.


\begin{proposition}
    \label{thm:Hausdorff-equivalent-properties}
    The following are equivalent for a topological space $X$:
    %
    \begin{enumprop}
        \item \label{enum:T2-space} $X$ is Hausdorff,
        \item \label{enum:T2-limits_unique} limits of nets (and hence of filters) in $X$ are unique, and
        \item \label{enum:T2-closed_diagonal} the diagonal $\Delta = \set{(x,x)}{x \in X}$ is closed in $X \prod X$.
    \end{enumprop}
\end{proposition}

\begin{proof}
\begin{proofsec}
    \item[\subcref{enum:T2-space} $\implies$ \subcref{enum:T2-limits_unique}]
    Let $(x_\alpha)_{\alpha \in A}$ be a net in $X$, and assume that $x_\alpha \to x$ and $x_\alpha \to y$. For every pair of neighbourhoods $U$ of $x$ and $V$ of $y$, $(x_\alpha)$ is eventually in $U \intersect V$. Hence $x$ and $y$ have no pair of disjoint neighbourhoods, so $x = y$.

    \item[\subcref{enum:T2-limits_unique} $\implies$ \subcref{enum:T2-closed_diagonal}]
    If $\Delta$ were not closed, then there would exist a net $(x_\alpha)_{\alpha \in A}$ in $X$ such that $(x_\alpha, x_\alpha) \to (x,y)$ where $x \neq y$, so the limit of $(x_\alpha)$ would not be unique.

    \item[\subcref{enum:T2-closed_diagonal} $\implies$ \subcref{enum:T2-space}]
    Let $x,y \in X$ be distinct points so that $(x,y) \not\in \Delta$. If $\Delta$ is closed, then $(x,y)$ has a neighbourhood $U \prod V$ in $X \times X$ disjoint from $\Delta$. But then $U$ and $V$ are disjoint neighbourhoods of $x$ and $y$ respectively, so $X$ is Hausdorff.
\end{proofsec}
\end{proof}


\section{Operations on Hausdorff spaces}

\begin{proposition} % TODO finish writing this up
    Let $X$ carry the initial topology induced by maps $f_\alpha \colon X \to X_\alpha$ for $\alpha \in A$. Assume that each $X_\alpha$ is Hausdorff, and that the $f_\alpha$ separate points in $X$. Then $X$ is also Hausdorff.

    In particular, any subspace of a $T_0$-space is $T_0$, and a product of $T_0$-spaces is $T_0$. Furthermore, if a nonempty product space is $T_0$, then every factor is $T_0$.
\end{proposition}

\begin{proof}
    TODO
\end{proof}


\section{Further properties of Hausdorff spaces}

\begin{proposition}
    \label{thm:Hausdorff-separate-compacts}
    In a Hausdorff space, disjoint compact sets can be separated.
\end{proposition}

\begin{proof}
    Let $K_1$ and $K_2$ be disjoint compact sets in a Hausdorff space $X$, and fix a point $x \in K_1$. Since $X$ is Hausdorff, $x$ can be separated from every $y \in K_2$. It follows from \cref{thm:separating_from_compacts} that $x$ can be separated from $K_2$. But then $K_2$ can be separated from every point in $K_1$, so another application of \cref{thm:separating_from_compacts} yields the desired claim.
\end{proof}


\begin{proposition}
    If $f,g \colon X \to Y$ are continuous and $Y$ is Hausdorff, then the set $\{f = g\} = \set{x \in X}{f(x) = g(x)}$ is closed. In particular, if $f$ and $g$ agree on a dense subset of $X$, then $f = g$.
\end{proposition}

\begin{proof}
    Let $x \in X$ be such that $f(x) \neq g(x)$. Since $Y$ is Hausdorff, $f(x)$ and $g(x)$ have disjoint neighbourhoods $U$ and $V$ respectively. Then $f\preim(U) \intersect g\preim(V)$ is a neighbourhood of $x$ on which $f$ and $g$ differ. Thus $\{f \neq g\}$ is open which proves the claim.

    Alternatively we may argue using nets\footnote{We refrain from using nets (or filters) as far as possible, or at least also provide proofs that do not depend on them. In this case nets do in fact clarify the necessity of the Hausdorff assumption, so we also include a proof using nets.}: Let $(x_\alpha)_{\alpha \in A}$ be a net in $\{f = g\}$ such that $x_\alpha \to x$. By continuity we have $f(x_\alpha) \to f(x)$ and $g(x_\alpha) \to g(x)$, and since limits are unique in $Y$ by \cref{enum:T2-limits_unique} and $f(x_\alpha) = g(x_\alpha)$ we have $f(x) = g(x)$.
\end{proof}



\chapter[Regular and T3-spaces][Regular and $T_3$-spaces]{Regular and $T_3$-spaces}

\section{Definition and equivalent properties}

\begin{definition}
    A topological space $X$ is \emph{regular} if, for every point $x \in X$ and closed subset $A \subseteq X$ with $x \not\in A$, $x$ has a neighbourhood $U$ and $A$ a neighbourhood $V$ with $U \intersect V = \emptyset$.

    If furthermore $X$ is $T_1$, then $X$ is said to satisfy the $T_3$ axiom and is called a \emph{$T_3$-space}.
\end{definition}
%
Notice that a regular space is \emph{not} necessarily Hausdorff since singletons are not closed. Of course a $T_3$-space is Hausdorff.


\begin{proposition}
    \label{thm:regular-equivalent-properties}
    A topological space $X$ is regular if and only if every $x \in X$ has a neighbourhood basis of closed sets.
\end{proposition}

\begin{proof}
\begin{proofsec}
    Assume that $X$ is regular, and let $U$ be an open neighbourhood of $x \in X$. Then $U^c$ is closed, so there exist disjoint open sets $V$ and $W$ with $x \in V$ and $U^c \subseteq W$. Then $x \in V \subseteq W^c \subseteq U$, so $W^c$ is the desired closed neighbourhood.

    Conversely, let $x \in X$ and $A \subseteq X$ closed with $x \not\in A$. Then $A^c$ is an open neighbourhood of $x$, so $A^c$ contains a closed neighbourhood $B$ of $X$. Then $\interior{B}$ and $B^c$ are disjoint open neighbourhoods of $x$ and $A$ respectively.
\end{proofsec}
\end{proof}


\section{Operations on regular spaces}

\begin{proposition}
    \label{thm:regular-initial-topology}
    Let $X$ carry the initial topology induced by maps $f_\alpha \colon X \to X_\alpha$ for $\alpha \in A$. Assume that each $X_\alpha$ is regular. Then $X$ is also regular.

    In particular, any subspace of a regular space is regular, and a product of regular spaces is regular. Furthermore, if a nonempty product space is regular, then every factor is regular.
\end{proposition}
%
Notice that we do \emph{not} require that the $f_\alpha$ separate points in $X$ since we do not need to distinguish individual points. If the $f_\alpha$ do separate points in $X$, then by \cref{thm:T1-initial-topology} the above also holds with \enquote{regular} replaced with \enquote{$T_3$}.

\begin{proof}
    Assume that each $X_\alpha$ is regular. We prove that each $x \in X$ has a neighbourhood basis of closed sets in accordance with \cref{thm:regular-equivalent-properties}, so let $U$ be an open neighbourhood of $x$. Then $x$ lies in some basic neighbourhood $\bigintersect_{i=1}^n f_{\alpha_i}\preim(U_{\alpha_i}) \subseteq U$, where $U_{\alpha_i}$ is open in $X_{\alpha_i}$. Hence $f_{\alpha_i}(x) \in U_{\alpha_i}$, and since $X_{\alpha_i}$ is regular $f_{\alpha_i}(x)$ has a closed neighbourhood $F_{\alpha_i}$ contained in $U_{\alpha_i}$. But then $f_{\alpha_i}\preim(F_{\alpha_i}) \subseteq f_{\alpha_i}\preim(U_{\alpha_i})$ is a closed neighbourhood of $x$, and finally $\bigintersect_{i=1}^n f_{\alpha_i}\preim(F_{\alpha_i})$ is a closed neighbourhood of $x$ contained in $U$ as desired.

    The final claim follows as in the proof of \cref{thm:T0-initial-topology} since each factor is homeomorphic to a subspace of the product.
\end{proof}


\begin{proposition}
    % \label{thm:regular-disjoint-union}
    Let $(X_\alpha)_{\alpha \in A}$ be a collection of regular spaces. Then the disjoint union $X = \coprod_{\alpha \in A} X_\alpha$ is also regular. % TODO: Again iff, right?
\end{proposition}
%
As with \cref{thm:T0-disjoint-union} we may immediately generalise this result to a larger class of final topologies: Namely those coinduced by maps $(f_\alpha)_{\alpha \in A}$ that are both open and closed, and that cover $X$. Note that this assumption is weaker than the previous assumptions, just as the assumptions in \cref{thm:regular-initial-topology} were weaker than those of \cref{thm:T0-initial-topology}. Again I am not aware of any interesting applications of this fact.

\begin{proof}
    Let $y \in X$, and let $U$ be a neighbourhood of $y$. There is then an $\alpha \in A$ and an $x \in X_\alpha$ such that $\iota_\alpha(x) = y$. Then $\iota_\alpha\preim(U)$ is an open neighbourhood of $x$, so $x$ has a closed neighbourhood $F$ contained therein. It follows that $y \in \iota_\alpha(F) \subseteq U$, and $\iota_\alpha(F)$ is a closed neighbourhood of $y$ since $\iota_\alpha$ is open and closed.
\end{proof}


\section{Further properties of regular spaces}

\begin{proposition}
    In a regular space, compact sets can be separated from disjoint closed sets.
\end{proposition}

\begin{proof}
    This is a direct consequence of \cref{thm:separating_from_compacts}.
\end{proof}


\begin{proposition}
    \label{thm:regular-Lindelof-is-paracompact}
    A regular Lindelöf space is paracompact.
\end{proposition}

\begin{proof}
    Let $X$ be a regular Lindelöf space, and let $\calU$ be an open cover of $X$. For every $x \in X$ pick a $U_x \in \calU$ with $x \in U_x$. By regularity $x$ has a neighbourhood $V_x$ such that $\closure{V}_{\!\!x} \subseteq U_x$. Then $\set{V_x}{x \in X}$ is also an open cover of $X$, so it has countable subcover $\set{V_{x_n}}{n \in \naturals}$ since $X$ is Lindelöf.

    For $n \in \naturals$ define sets
    %
    \begin{equation*}
        W_n = U_{x_n} \setminus \bigunion_{k < n} \closure{V}_{\!\!x_k}.
    \end{equation*}
    %
    For $x \in X$ there is a smallest $k \in \naturals$ such that $x \in \closure{V}_{\!\!x_k}$, so $x \in W_k$. Hence $\set{W_n}{n \in \naturals}$ is also an open cover of $X$, and it is clearly a refinement of $\calU$. It is also locally finite, since $x \in V_{x_k}$ for some $k \in \naturals$, but $V_{x_k}$ does not intersect $W_n$ for $n > k$. Hence $X$ is paracompact.
\end{proof}


\section{Conditions for regularity}

\begin{corollary}
    Pseudometric spaces are regular.
\end{corollary}

\begin{proof}
    This will follow from \cref{thm:pseudometric-completely-regular} since completely regular spaces are regular.
\end{proof}


\begin{corollary}
    Locally compact Hausdorff spaces are regular.
\end{corollary}

\begin{proof}
    This will follow from \cref{thm:Urysohn-LCH} since completely regular spaces are regular.

    Alternatively, by \cref{thm:LCH-equivalent-condition} every point of an LCH space has a neighbourhood basis of compact (and hence closed) sets, which implies regularity by \cref{thm:regular-equivalent-properties}.
\end{proof}



\chapter{Completely regular and Tychonoff spaces}

\section{Definition and equivalent properties}

\begin{definition}
    A topological space $X$ is \emph{completely regular} if, for every point $x \in X$ and closed subset $A \subseteq X$ with $x \not\in A$, there is a continuous function $f \colon X \to [0,1]$ with $f(x) = 0$ and $f(A) = 1$. Such a function is said to \emph{separate} $x$ and $A$.

    If furthermore $X$ is $T_1$, then $X$ is said to satisfy the $T_{3\frac{1}{2}}$-axiom and is called \emph{Tychonoff}.
\end{definition}
%
A completely regular space is indeed regular: If $f$ separates $x$ and $A$, then $f\preim([0,1/2))$ and $f\preim((1/2,1])$ are disjoint neighbourhoods of $x$ and $A$ respectively.

We now prove that a space is completely regular precisely when the bounded continuous functions on the space induce the topology. Of course, these functions are already continuous, so this says that there are \emph{enough} continuous functions for them to characterise the topology.

To prove this we take a small detour by studying the defining property of completely regular spaces in greater generality. We say that a collection $(f_\alpha)_{\alpha \in A}$ of functions $f_\alpha \colon X \to X_\alpha$ between topological spaces \emph{separates points from closed sets} if whenever $C \subseteq X$ is closed and $x \not\in C$, then $f_\alpha(x) \not\in \closure{f_\alpha(C)}$ for some $\alpha \in A$.

\begin{proposition}
    \label{thm:separating-points-from-closed-sets-basis}
    A collection $(f_\alpha)_{\alpha \in A}$ of functions $f_\alpha \colon X \to X_\alpha$ between topological spaces separates points from closed sets if and only if the sets $f_\alpha\preim(V)$, for $\alpha \in A$ and $V \subseteq X_\alpha$ open, form a basis for the topology on $X$.
\end{proposition}

\begin{proof}
    First assume that $(f_\alpha)$ separates points from closed sets, let $U \subseteq X$ be open and let $x \in U$. Then $U^c$ is closed, so there is some $\alpha \in A$ such that $f_\alpha(x) \not\in \closure{f_\alpha(U^c)}$. Then
    %
    \begin{equation*}
        U^c
            \subseteq f_\alpha\preim \bigl( f_\alpha(U^c) \bigr)
            \subseteq f_\alpha\preim \bigl( \closure{f_\alpha(U^c)} \bigr).
    \end{equation*}
    %
    So letting $V = \closure{f_\alpha(U^c)}^c$ we find that $x \in f_\alpha\preim(V) \subseteq U$ as desired.

    Conversely, assume that the sets $f_\alpha\preim(V)$ form a basis for the topology on $X$. Let $x \in X$ and $C \subseteq X$ closed with $x \not\in C$. There is an $\alpha \in A$ and an open $V \subseteq X_\alpha$ such that $x \in f_\alpha\preim(V) \subseteq C^c$. Then $V$ is a neighbourhood of $f_\alpha(x)$ disjoint from $f_\alpha(C)$, so $f_\alpha(x) \not\in \closure{f_\alpha(C)}$.
\end{proof}


\begin{corollary}
    \label{thm:separating-points-from-closed-sets-weak-topology}
    If $(f_\alpha)_{\alpha \in A}$ is a collection of functions $f_\alpha \colon X \to X_\alpha$ between topological spaces which separates points from closed sets, then $X$ carries the weak topology induced by the maps $f_\alpha$.
\end{corollary}

\begin{proof}
    \cref{thm:separating-points-from-closed-sets-basis} shows that the collection of preimages $f_\alpha\preim(V)$, for $\alpha \in A$ and $V \subseteq X_\alpha$ open, forms a basis for the topology on $X$, so it in particular generates the topology.
\end{proof}



\begin{theorem}
    \label{thm:completely-regular-weak-topology}
    A topological space $X$ is completely regular if and only if it has the weak topology induced by $C_b(X)$.
\end{theorem}

\begin{proof}
    If $X$ is completely regular, then $C_b(X)$ separates points from closed sets by definition, so \cref{thm:separating-points-from-closed-sets-weak-topology} shows that $X$ carries the the weak topology induced by $C_b(X)$.

    Conversely, suppose that $X$ has the weak topology induced by $C_b(X)$. Let $U \subseteq X$ be open, and let $x \in U$. Then there are functions $f_1, \ldots, f_n \in C_b(X)$ and subbasic open sets $V_1, \ldots, V_n \subseteq \reals$ such that
    %
    \begin{equation*}
        x
            \in \bigintersect_{i=1}^n f_i\preim(V_i)
            \subseteq U.
    \end{equation*}
    %
    By changing the sign on the $f_i$ if necessary, we may assume that each $V_i$ is on the form $(a_i, \infty)$. Define functions $g_i \colon X \to \reals$ by $g_i(x) = (f_i(x) - a_i) \join 0$. Then $g_i\preim(0,\infty) = f_i\preim(a_i,\infty)$, so
    %
    \begin{equation*}
        x
            \in \bigintersect_{i=1}^n g_i\preim(0,\infty)
            \subseteq U.
    \end{equation*}
    %
    %
    Let $g = g_1 g_2 \cdots g_n$. Then $g(x) > 0$, so $x \in g\preim(0,\infty)$. Furthermore, if $g(y) > 0$ then each $g_i(y) > 0$. it follows that
    %
    \begin{equation*}
        x
            \in g\preim(0,\infty)
            \subseteq U.
    \end{equation*}
    %
    Then $g(x) \neq 0$, but $g(U^c) = 0$, so $X$ is completely regular.
\end{proof}


% \begin{proposition}
%     A topological space is Tychonoff if and only if it is homeomorphic to a subspace of a cube, i.e. a product of compact intervals.
% \end{proposition} % Leave this out?

% \begin{proof}
    
% \end{proof}


\section{Conditions for complete regularity}

\begin{proposition}
    \label{thm:pseudometric-completely-regular}
    Pseudometric spaces are completely regular.
\end{proposition}
%
In \cref{thm:metric-space-normal} we will see that pseudometric spaces are also normal, but since a pseudometric space is not necessarily $T_1$, this does not imply that it is (completely) regular. Hence the necessity of the present proposition.

\begin{proof}
    Let $(S,\rho)$ be a pseudometric space, $x \in S$, and let $A \subseteq S$ be closed with $x \not\in A$. Since $A$ is closed, the map $y \mapsto \rho(y,A)$ is zero on $A$ and nonzero at $y$, and it is continuous by \cref{thm:distance-to-set}.
\end{proof}


We now wish to show that locally compact Hausdorff spaces are completely regular. In the presence of the Hausdorff axiom, complete regularity is weaker than normality, and \emph{compact} spaces are normal, so it is perhaps not surprising that \emph{locally} compact spaces are completely regular.

To show this we will prove a version of Urysohn's lemma for locally compact Hausdorff spaces. This relies on the Urysohn lemma for normal spaces covered in the next section, but we place this discussion here since we are interested in it in the context of completely regular spaces.


\begin{theorem}[Urysohn's Lemma, locally compact version]
    \label{thm:Urysohn-LCH}
    Let $X$ be a locally compact Hausdorff space, and let $K \subseteq U \subseteq X$ with $K$ compact and $U$ open. Then there exists a continuous function $f \colon X \to [0,1]$ such that $f(K) = 1$ and $f$ vanishes outside a compact subset of $U$.
\end{theorem}

\begin{proof}
    By \cref{thm:LCH-compact-set-has-compact-nhood} there is a precompact open set $V$ with $K \subseteq V \subseteq \closure{V} \subseteq U$. Since compact Hausdorff spaces are normal, we can apply Urysohn's lemma for normal spaces to $\closure{V}$: This yields a continuous function $f \colon \closure{V} \to [0,1]$ with $f(K) = 1$ and $f(\boundary V) = 0$. Extend $f$ to $X$ by letting $f(\closure{V}^c) = 1$.

    We claim that $f$ is continuous on $X$. Let $B \subseteq [0,1]$ be closed. If $0 \not\in B$, then $f\preim(B) = (f|_{\closure{V}})\preim(B)$ is closed in $\closure{V}$, hence also in $X$. On the other hand, if $0 \in B$, then
    %
    \begin{equation*}
        f\preim(B)
            = (f|_{\closure{V}})\preim(B) \union \closure{V}^c
            = (f|_{\closure{V}})\preim(B) \union V^c,
    \end{equation*}
    %
    where the last equality follows since $\boundary V \subseteq (f|_{\closure{V}})\preim(B)$. Again $f\preim(B)$ is closed, so $f$ is continuous.
\end{proof}


\begin{corollary}
    Locally compact Hausdorff spaces are completely regular, hence Tychonoff.
\end{corollary}

\begin{proof}
    Let $X$ be a locally compact Hausdorff space, let $x \in X$ and $A \subseteq X$ be a closed subset. In the notation of Urysohn's lemma, let $K = \{x\}$ and $U = A^c$, which yields a continuous function $f \colon X \to [0,1]$ with $f(x) = 1$ and $f(A) = 0$.
\end{proof}


\section{Further properties of completely regular spaces}

% For the purposes of stating and proving the next theorem, we write $C_b(X,\calT)$ for the space of bounded real-valued functions on $X$ that are continuous with respect to the topology $\calT$ on $X$.

\begin{proposition}
    Let $X$ be a topological space. There exists a Tychonoff space $Y$ such that $C_b(X)$ and $C_b(Y)$ are isomorphic as rings.
\end{proposition}
%
Hence, if one is interested in studying rings of bounded functions, then one may as well assume that the domain is Tychonoff.

\begin{proof}
    Let $X'$ be $X$ equipped with the weak topology induced by $C_b(X)$. Then since replacing the topology with a weaker one does not introduce any new continuous functions, we have $C_b(X) = C_b(X')$. Hence $X'$ is completely regular by \cref{thm:completely-regular-weak-topology}.

    Now consider the $T_0$-identification $X'/{\equiv}$ of $X'$. By \cref{thm:T0-identification-continuous-functions} we have $C(X') \cong C(X'/{\equiv})$, and this isomorphism clearly restricts to an isomorphism $C_b(X') \cong C_b(X'/{\equiv})$, proving the claim.
\end{proof}



\chapter[Normal and T4-spaces]{Normal and $T_4$-spaces}

\section{Definition}

\begin{definition}
    A topological space $X$ is \emph{normal} if, for every pair of disjoint closed subsets $A,B \subseteq X$, $A$ has a neighbourhood $U$ and $B$ a neighbourhood $V$ with $U \intersect V = \emptyset$.

    If furthermore $X$ is $T_1$, then $X$ is said to satisfy the $T_4$ axiom and is called a \emph{$T_4$-space}.
\end{definition}

We discuss conditions that are equivalent to normality in our discussion of Urysohn's lemma below.


\section{Conditions for normality}

\begin{proposition}
    \label{thm:metric-space-normal}
    Pseudometric spaces are normal.
\end{proposition}

\begin{proof}
    Let $(S,\rho)$ be a pseudometric space, and let $A, B \subseteq S$ be disjoint closed subsets. For $a \in A$ let $r_a = \rho(a,B)/2 > 0$, and for $b \in B$ let $r_b = \rho(b,A)/2 > 0$. Let
    %
    \begin{equation*}
        U = \bigunion_{a \in A} B(a,r_a)
        \quad \text{and} \quad
        V = \bigunion_{b \in B} B(b,r_b).
    \end{equation*}
    %
    We claim that $U$ and $V$ are disjoint. Let $x \in U$ and $y \in V$. Then $x \in B(a,r_a)$ and $y \in B(b,r_b)$ for some $a \in A$ and $b \in B$. Then
    %
    \begin{equation*}
        \rho(a,b)
            \leq \rho(a,x) + \rho(x,y) + \rho(y,b)
            < \rho(x,y) + r_a + r_b,
    \end{equation*}
    %
    which implies that
    %
    \begin{equation*}
        0
            \leq \rho(a,b) - r_a - r_b
            < \rho(x,y),
    \end{equation*}
    %
    where the first inequality follows since
    %
    \begin{equation*}
        \rho(a,b)
            = \frac{\rho(a,b) + \rho(a,b)}{2}
            \geq \frac{\rho(a,B)}{2} + \frac{\rho(b,A)}{2}
            = r_a + r_b.
    \end{equation*}
\end{proof}



\begin{proposition}
    \label{thm:paracompact-Hausdorff-is-normal}
    Every paracompact Hausdorff space is normal, hence $T_4$.
\end{proposition}
% Hausdorff can be slightly weakened. Cool link: https://math.stackexchange.com/questions/1315614/r-1-paracompact-spaces-are-normal
% Also: https://en.wikipedia.org/wiki/Separation_axiom
%
In particular, every \emph{compact} Hausdorff space is $T_4$. This also follows easily since two disjoint compact sets can be separated in a Hausdorff space by \cref{thm:Hausdorff-separate-compacts}.

\newcommand{\bbU}{\mathbb{U}}
\newcommand{\bbV}{\mathbb{V}}

\begin{proof}
    Let $X$ be a paracompact Hausdorff space, $A$ a closed subset and $q \in X \setminus A$. For every $p \in A$ there exist, by the Hausdorff assumption, disjoint open neighbourhoods $U_p$ and $V_p$ of $p$ and $q$ respectively. Each $p \in A$ thus has a neighbourhood $U_p$ such that $q \not\in \closure{U}_p$.
    
    The sets $U_p$ is an open cover of $A$, so by paracompactness of $A$ (cf. \cref{thm:compact-Lindelof-closed-subset}) we obtain a locally finite subcover $\calU$. Letting $\bbU = \bigunion_{U \in \calU} U$ and $\bbV = X \setminus \closure{\bbU}$ we then have two disjoint open sets, and by local finiteness of $\calU$ we have $\closure{\bbU} = \bigunion_{U \in \calU} \closure{U}$, so $\bbV$ contains $q$.

    This shows that $X$ is regular. The same argument then shows that $X$ is normal, by using regularity instead of the Hausdorff property.
\end{proof}


\begin{proposition}
    A regular Lindelöf space is normal.
\end{proposition}

\begin{proof}
    Let $X$ be a regular Lindelöf space, and let $A,B \subseteq X$ be disjoint closed subsets. By regularity, every $a \in A$ has a neighbourhood $U_a$ such that $\closure{U_a} \intersect B = \emptyset$. Similarly, every $b \in B$ has a neighbourhood $V_b$ separating it from $A$. Since $A$ and $B$ are themselves Lindelöf by \cref{thm:compact-Lindelof-closed-subset}, they are covered by countably many $U_a$ and $V_b$ respectively, say $A \subseteq \bigunion_{n \in \naturals} U_n$ and $B \subseteq \bigunion_{n \in \naturals} V_n$.

    Now define sequences of sets $S_n$ and $T_n$ by
    %
    \begin{equation*}
        S_n = U_n \setminus \closure{ \bigunion_{i < n} T_i}
        \quad \text{and} \quad
        T_n = V_n \setminus \closure{ \bigunion_{i \leq n} S_i}.
    \end{equation*}
    %
    (Notice the strict and non-strict inequalities.) Define the sets $S = \bigunion_{n \in \naturals} S_n$ and $T = \bigunion_{n \in \naturals} T_n$. Clearly $S$ is a neighbourhood of $A$ and $T$ of $B$. We claim that they are also disjoint: Let $x \in S_n$ for some $n \in \naturals$. Then $x \not\in T_m$ for $m < n$ by the definition of $S_n$, and $x \not\in T_m$ for $m \geq n$ by the definition of $T_m$. 
\end{proof}

We thus find that compact Hausdorff implies normal, and that regular Lindelöf implies normal. That is, normality follows from a compactness property along with a separability property. We cannot weaken compactness to Lindelöf and still have normality, but also \enquote{strengthening} Hausdorff to regularity retains normality. (Of course, it is not regularity but $T_3$ that is stronger than Hausdorff.)


\section{Urysohn's Lemma and related results}

If $X$ is a topological space and $A,B \subseteq X$ are closed sets, then a continuous function $f \colon X \to [0,1]$ with $f(A) = 0$ and $f(B) = 1$ is called a \emph{Urysohn function} for $A$ and $B$.

\begin{theorem}[Urysohn's Lemma]
    A topological space $X$ is normal if and only if there is a Urysohn function for every pair of closed subsets of $X$.
\end{theorem}

\begin{proof}
    First assume that $X$ is normal and that $A,B \subseteq X$ are closed. By normality there is an open set $U_{1/2}$ such that
    %
    \begin{equation*}
        A
            \subseteq U_{1/2}
            \subseteq \closure{U}_{\!1/2}
            \subseteq B^c.
    \end{equation*}
    %
    Then $A$ and $U_{1/2}^c$ are disjoint closed sets, and so are $\closure{U}_{\!1/2}$ and $B$. Hence there exist open sets $U_{1/4}$ and $U_{3/4}$ such that
    %
    \begin{equation*}
        A
            \subseteq U_{1/4}
            \subseteq \closure{U}_{\!1/4}
            \subseteq U_{1/2}
            \subseteq \closure{U}_{\!1/2}
            \subseteq U_{3/4}
            \subseteq \closure{U}_{\!3/4}
            \subseteq B^c.
    \end{equation*}
    %
    Let $\Delta$ be the set of all dyadic rational numbers\footnote{Recall that a dyadic rational number is a number on the form $p/2^n$ for $p \in \ints$ and $n \in \naturals$.} in $(0,1)$. We may thus recursively define for every $r \in \Delta$ a set $U$ with the following properties:
    %
    \begin{enumerate} % Proof enumerate??
        \item $A \subseteq U_r$ and $\closure{U}_{\!r} \subseteq B^c$ for each $r \in \Delta$, and

        \item \label{enum:Urysohn-proof-closure-inside-open} $\closure{U}_{\!r} \subseteq U_s$ if $r < s$, for $r,s \in \Delta$.
    \end{enumerate}
    %
    We furthermore let $U_1 = X$. Then define a function $f \colon X \to [0,1]$ by $f(x) = \inf \set{r}{x \in U_r}$. Since $A \subseteq U_r \subseteq B^c$ for all $r \in \Delta$, we clearly have $f(A) = 0$ and $f(B) = 1$, and that $0 \leq f(x) \leq 1$ for all $x \in X$.

    It remains to be shown that $f$ is continuous. Let $\alpha \in \reals$ and $x \in X$, and notice that $f(x) < \alpha$ if and only if $x \in U_r$ for some $r < \alpha$, which is true just when $x \in \bigunion_{r < \alpha} U_r$. Hence,
    %
    \begin{equation*}
        f\preim((-\infty,\alpha))
            = \bigunion_{r < \alpha} U_r
    \end{equation*}
    %
    is open. Similarly $f(x) > \alpha$ if and only if $x \not\in U_r$ for some $r > \alpha$, which is equivalent to $x \not\in \closure{U}_{\!s}$ for some $s > \alpha$ by property \cref{enum:Urysohn-proof-closure-inside-open} above. This is the case if and only if $x \in \bigunion_{s > \alpha} (\closure{U}_{\!s})^c$. It follows that
    %
    \begin{equation*}
        f\preim((\alpha,\infty))
            = \bigunion_{s > \alpha} (\closure{U}_{\!s})^c
    \end{equation*}
    %
    is also open. Hence $f$ is continuous.

    Conversely, assume that $f$ is a Urysohn function for a pair of disjoint closed sets $A,B \subseteq X$. Then $f\preim([0,1/2))$ and $f\preim((1/2,1])$ are disjoint neighbourhoods of $A$ and $B$ respectively, so $X$ is normal.
\end{proof}


\begin{theorem}[The Tietze extension theorem]
    \label{thm:Tietze-extension}
    A topological space $X$ is normal if and only if any continuous function $f \colon A \to \reals$ on a closed set $A \subseteq X$ can be extended to a continuous function on all of $X$, i.e. there exists a continuous $F \colon X \to \reals$ such that $f = F|_A$.

    Furthermore, if $a,b \in \reals$ and $f(A) \subseteq [a,b]$ then $F$ can be chosen such that $F(X) \subseteq [a,b]$.
\end{theorem}

\begin{proof}
    
\end{proof}


\nocite{*}

\printbibliography


\end{document}
